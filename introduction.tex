\section{Introduction}
Database systems increasingly use materialization, caching a computed
query result, to speed up queries on large datasets {[}?{]}. 
However, as the base tables update, materialized views derived from these tables
become increasingly stale. Incrementally keeping the views up-to-date,
also called incremental maintenance, has been well studied {[}?{]}
including a variety of techniques such as batch maintenance {[}?{]}
and lazy maintenance {[}?{]}. 

Unfortunately, in many desired applications, incremental view maintenance
can be very costly. 
When the base tables change rapidly, the database has to propagate these rapid changes to
all of the materialized views.  
This cost breaks down into two components: applying
the view definition to the updates, and then writing the ``delta''
view to the out-of-date view. These two pieces can be costly in different
applications. (1) In distributed environments where the view is partitioned
over a cluster, incremental view maintenance often neccesitates communicating
the delta view. (2) Systems such as Apache Spark, Cloudera Impala,
and Apache Tez {[}?{]} offer materialized view support, however, are
not optimized for selective updates nor have native support for indices.
This can lead to high maintenance costs in applications where the
views are derived from joins that are not aligned with the partitioning
of the base tables. (3) Base data is often raw requiring pre-processing
such as string processing, deserialization, and formatting; all of
which can can be expensive to run on a large number of updates. 

Consequently, a commonly applied approach is to maintain the views periodically and
schedule maintenance at less active times eg. nightly; while accepting
that in the interim results will be stale. This approach avoids creating
an undue bottleneck of maintanance for each udpate and reduces resource contention during high traffic times; 
however a user querying the system can get results that are stale.
Furthermore, in general, the staleness is unbounded and user does not know how the staleness affects their query results.

Querying a stale view is similar to problems studied in data cleaning{[}?{]}.
When databases are dirty, query results can be arbitrarily wrong.
Data cleaning is used to remove data errors but this can be very costly either 
requiring machine learning to classify errors or even human intervention.
SampleClean is a query processing framework that answers aggregate
queries on dirty datasets by applying potentially expensive cleaning
techniques to just a sample. The results, while approximate, are bounded
with respect to the clean data and the system offers a flexible tradeoff
between cleaning cost and result accuracy. Similarly, a stale row
and an expensive incremental maintenance scheme, mirrors the problem
setting studied in SampleClean. 

In this paper, we propose a data cleaning approach for approximate,
bounded aggregation queries on stale views. Instead of maintaining
the entire view, we maintain only a small sample of the view. Then
given an aggregation query on this view, from this small sample, we
can estimate how the updates affect the query result. We apply this
estimate to correct the dirty aggregation query result on the stale
data. We call this approach \emph{approximate query correction}. 
These corrections are provably bounded, in contrast to the unbounded stalness,
and the sampling gives a flexible tradeoff to meet performance constraints such as throughput.
Sampling helps reduces both bottlenecks in view maintenance, delta
view calculation and view updating, as it reduces the number of updates
that need to processed and then written.

Another relevant concept from data cleaning is outlier detection {[}?{]}.
Sampling has the potential to mask outliers and in fact it is known
heavy-tailed distributions are poorly approximated from samples {[}?{]}.
Recent work has shown that outlier indexing, ie. separating the values
from the tail, can improve sample estimates in such distributions.
Coupling outlier detection with sampled views has an interesting implication;
not only are the outliers themselves interesting for analysis, but
the information from the outliers can potentially improve query accuracy
or likewise reduce the number of needed samples.
In this work, we propose an outlier indexing framework that guarantees that
rows in the materialized view derived from an ``outlier" record, one with
an abnormal attribute value, are included in the sample.

In summary, our contributions are as follows:
\begin{itemize}
\item We present a query processing framework that corrects aggregation queries on stale
views using a sample of up-to-date data.
\item We couple this approach with an oultier indexing framework that allows
for selection queries on outliers and we show both analytically and empirically that 
this can improve query accuracy.
\item We evaluate our approach on two systems, Apache SparkSQL and MySQL,
and discuss how the different systems affect performance performance
parameters.
\end{itemize}
\section{Introduction}
Materialized views, stored pre-computed query results, are used to facilitate fast query processing on large datasets \cite{gupta1995maintenance, chirkova2011materialized, halevy2001answering}.
Materialization and related concepts such as selecting queries to materialize
have been well studied in recent research \cite{zaharia2012resilient,lefevre2014opportunistic, bailis2014scalable, perez2014history}.
This research has further expanded beyond the SQL setting \cite{nikolic2014linview, zhang2014mat} and 
has shown promising results when applied to numerical linear algebra and machine learning.
However, when derived from frequently changing tables,
materialized views face the challenge of \emph{staleness} where the pre-computed results need to be updated.
To avoid expensive recalculation, incrementally updating the views,
also called incremental maintenance, has been well studied \cite{gupta1995maintenance, chirkova2011materialized}.

Unfortunately, in many applications, incremental maintenance on each update can be very costly. 
Consequently, it is common to defer maintenance to a later time \cite{chirkova2011materialized, zhou2007lazy}.
Deferral allows for many advantages such as batching updates together to amortize overheads and scheduling updates at times when there are more system resources available, for example, at night.
Deferral allows for increased flexibility to meet the resource constraints of the system, but may not guarantee that the view will be up-to-date.

%When a materialized view is out-of-date during a maintenance gap, queries issued to the view can have \emph{stale} results. 

In this work, we address the view-maintenance problem from a different perspective.
We explore whether refreshing stale rows in a materialized view can be modeled as a data cleaning problem and whether stale query results can be \emph{cleaned}.
While much of the data cleaning literature focuses on improving query accuracy on dirty datasets,
recent work has also considered the costs of data cleaning \cite{wang1999sample}.
Recent results show that to answer aggregate queries, such as SUM, COUNT, and AVG, on dirty datasets, it suffices to clean a small, representative sample of dirty records.
This model raises a new possibility for materialized views, namely, that a sample of up-to-date rows can be used to answer aggregate queries without incurring the cost of full maintenance.

We propose Sample-View-Clean (SVC), a framework that uses a \emph{sample} of up-to-date data to \emph{clean} stale aggregation query results.
While approximate, the corrected query result can be bounded within confidence intervals.
This framework will be complementary to existing deferred maintenance approaches; when the materialized views are stale between maintenance cycles, we can apply SVC for approximately results for far less costs than having to maintain the entire view.
Without SVC, existing work allows the user to control the freshness of queries by changing maintenance parameters (e.g. nightly maintenance vs. hourly maintenance) based on prior experience.
Without bounds on the results, a burst of updates can lead to unexpected changes in query accuracy.
On the other hand, SVC gives results that are fresh and the user controls a bounded approximation error with the sampling ratio.

%Sampling gives the user the ability to scale the performance 
%Existing techniques allow the user to control the freshness of queries by chaning maintenance parameters (e.g. nightly maintenance vs. hourly maintenance) based on prior experience. However, without bounds on the results, a burst of updates can lead to unexpected changes in query accuracy.
%On the other hand, our approach gives results that are, in expectation, fresh and the user controls the tightness of the bound with the sampling ratio.

%The SampleClean project [?] studied a related problem of bounding aggregate queries on dirty datasets but did not consider materialized views or the effect of missing records.

SVC has three main components: (1) sampling, (2) correction, and (3) outlier indexing. In (1), we define an ``update pattern" which represents how an update affects the derived view, and we take a sample of these patterns. (2)  From the sample, we estimate how much the updates affect the query and we use this estimate to clean the stale query result.
Finally, in (3) sampling is known to be sensitive to outliers \cite{chaudhuri2001overcoming}.
We utilize a technique called outlier indexing \cite{chaudhuri2001overcoming}, which guarantees that rows in the materialized view derived from an ``outlier" record (one that has abnormal attribute values) is contained in the sample, which can be used to increase correction accuracy.

Our approach can be implemented with a relatively small overhead: at maintenance time the generation of random numbers to build the sample, and at query execution time single pass over a small sample of data to estimate a correction for the query.
Consequently, sampling can significantly save on maintenance costs and give a flexible tradeoff between accuracy and performance.
To summarize, our contributions are as follows:
\begin{itemize}
  \item We model the incremental maintenance problem as a data cleaning problem and staleness as a type of data error.
  \item We define the concept of ``update patterns", a logical unit that represents how an update affects a materialized view, and show how to sample the update patterns. We show analytically and empirically that sampling update patterns can be far more efficient than full incremental maintenance.
  \item Using a sample of update patterns, we can correct stale aggregation queries on materialized views. We bound these corrections in confidence intervals and prove optimality of our approach.
  \item We use an outlier index to increase the accuracy of the approach for power-law, long-tailed, and skewed distributions.
  \item We evaluate our approach on real and synthetic datasets in both single-node and distributed environments.
\end{itemize}

The paper is organized in the following way. 
In Section \ref{sec-background}, we introduce materialized views and discuss the current maintenance challenges.
Next, in Section \ref{sec-arch}, we give a brief overview of our overall system architecture.
In Section \ref{sampling} and \ref{correction}, we describe the sampling and query processing of our technique.
In Section \ref{outlier}, we describe the outlier indexing framework.
Then, in Section \ref{exp}, we evaluate our approach.
Finally, we end with our Related Work in Section \ref{related} and our Conclusions and Future Work in Section \ref{conclusion}.

\iffalse
 These two pieces can be costly in different
applications. (1) In distributed environments where the view is partitioned
over a cluster, incremental view maintenance often necessitates communicating
the delta view. (2) Systems such as Apache Spark, Cloudera Impala,
and Apache Tez {[}?{]} offer materialized view support, however, are
not optimized for selective updates nor have native support for indices.
This can lead to high maintenance costs in applications where the
views are derived from joins that are not aligned with the partitioning
of the base tables. (3) Base data is often raw requiring pre-processing
such as string processing, deserialization, and formatting; all of
which can can be expensive to run on a large number of updates. 



Querying a stale view is similar to problems studied in data cleaning{[}?{]}.
When databases are dirty, query results can be arbitrarily wrong.
Data cleaning is used to remove data errors but this can be very costly either 
requiring machine learning to classify errors or even human intervention.
SampleClean is a query processing framework that answers aggregate
queries on dirty datasets by applying potentially expensive cleaning
techniques to just a sample. The results, while approximate, are bounded
with respect to the clean data and the system offers a flexible tradeoff
between cleaning cost and result accuracy. Similarly, a stale row
and an expensive incremental maintenance scheme, mirrors the problem
setting studied in SampleClean. 

In this paper, we propose a data cleaning approach for approximate,
bounded aggregation queries on stale views. Instead of maintaining
the entire view, we maintain only a small sample of the view. Then
given an aggregation query on this view, from this small sample, we
can estimate how the updates affect the query result. We apply this
estimate to correct the dirty aggregation query result on the stale
data. We call this approach \emph{approximate query correction}. 
These corrections are provably bounded, in contrast to the unbounded stalness,
and the sampling gives a flexible tradeoff to meet performance constraints such as throughput.
Sampling helps reduces both bottlenecks in view maintenance, delta
view calculation and view updating, as it reduces the number of updates
that need to processed and then written.

Another relevant concept from data cleaning is outlier detection {[}?{]}.

In this work, we propose an outlier indexing framework that guarantees that
rows in the materialized view derived from an ``outlier" record, one with
an abnormal attribute value, are included in the sample.

In summary, our contributions are as follows:
\begin{itemize}
\item We present a query processing framework that corrects aggregation queries on stale
views using a sample of up-to-date data.
\item We couple this approach with an oultier indexing framework that allows
for selection queries on outliers and we show both analytically and empirically that 
this can improve query accuracy.
\item We evaluate our approach on two systems, Apache SparkSQL and MySQL,
and discuss how the different systems affect performance performance
parameters.
\end{itemize}
\fi

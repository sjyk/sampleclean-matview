\section{Delta-Algebra Optimizations}
As hinted in the previous section, optimizing joins is of particular interest. 
One of the most important use cases of materialized views to pre-compute expensive joins.
In \cite{DBLP:journals/vldb/KochAKNNLS14}, Koch et al. list the following properties of delta relations and joins:
\[ \Delta(R_1 \bowtie R_2) =  (R_1 \bowtie \Delta R_2) \dot{\cup} (\Delta R_1 \bowtie  R_2) \dot{\cup} (\Delta R_1 \bowtie  \Delta R_2)\]
\[ \nabla(R_1 \bowtie R_2) =  (R_1 \bowtie \nabla R_2) \dot{\cup} (\nabla R_1 \bowtie  R_2) \dot{\cup} (\nabla R_1 \bowtie  \nabla R_2)\]
When working with these delta relations, it would seem as though one join turns into three joins.
In this section, we explore when prune such expressions based on prior knowledge about the base relations and the updates.
\reminder{optimizations work only for insertions}

\subsection{Foreign-Key Constraints}

\reminder{the way that we model this subexpressions may violate constraints}

\textbf{Many-to-one insertion: } Suppose $R_1$ and $R_2$ have a foreign key constraint between them, and $R_1$ is the ``many" relation. Then, since every tuple in $R_1$ has to map to exactly one tuple in $R_2$, we know that insertions to $R_2$ could not affect existing tuples in $R_1$. Thus, we can prune the delta expression to:
\[ \Delta(R_1 \bowtie R_2) =  (\Delta R_1 \bowtie  R_2) \dot{\cup} (\Delta R_1 \bowtie  \Delta R_2)\]

\textbf{One-to-one insertion: } Suppose $R_1$ and $R_2$ have a foreign key constraint between them, and the relationship is exactly one-to-one. Then, since every tuple in $R_1$ has to map to exactly one tuple in $R_2$ and vice versa, we know that insertions to $R_2$ could not affect existing tuples in $R_1$ (and vice versa). Thus, we can prune the delta expression to:
\[ \Delta(R_1 \bowtie R_2) =  (\Delta R_1 \bowtie  \Delta R_2)\]

\subsection{Unique-Key Constraints and Stale Views}
In some cases, due to unique key constrains our delta expressions may be guaranteed to not overlap with another table. Consider the case of a simple selection view that selects records from a base relation. To update the view, we should take the union and not the join of the delta relation and the stale view. 
\[ \Delta(S) = S \dot{\cup} \Delta(S_{def}) \]
Recall our hash operator pushdown rules, we can push the operator to both S and the delta relation.
Thus, we can sample the stale view and the updates to the view independently.
Since S is already materialized, there is no need to sample S as that will not save any effort.
The implications of this optimization will be more evident in the following section about formulating a correction plan as this gives us the strong guarantee that our query time is strictly less than what it would be without sampling.

\subsection{Time/Causality Constraints}
In many important use cases such as log analysis, one or more of the base relations have time (e.g. timestamps or sequence numbers) as their primary key.
\textbf{$\theta$ join on time: } If we are applying a $\theta$ join, where the join key is time, then we can apply a similar optimization to the many-to-one insertion case.
\[ \Delta(R_1 \bowtie R_2) =   (\Delta R_1 \bowtie_{\theta}  R_2) \dot{\cup} (\Delta R_1 \bowtie_{\theta}  \Delta R_2)\]

\textbf{Aggregation on time: } Similarly, with group by aggregations if the group by expression is a monotonic function of time (e.g. count visits group by day), then we can apply a similar trick as used when there are unique key constraints and bound the effect of updating a stale view to the last group by key.




\documentclass{sig-alternate}

\usepackage{enumitem}
\usepackage{framed}
\usepackage{listings}
\usepackage{amstext}
\usepackage{amstext}
\usepackage{pdfpages}
\usepackage{alltt}
\usepackage{epstopdf}
\usepackage{xspace,colortbl}
\usepackage[USenglish]{babel}
\usepackage{multirow}
\usepackage{url}
\usepackage{subfigure}
\usepackage{graphicx}%%
\usepackage{amssymb}
\usepackage{fmtcount}
\usepackage{amsfonts}
\usepackage{xspace}
\usepackage{amsmath}
\usepackage{multirow}
\usepackage[mathscr]{eucal}
%\usepackage{psfrag}
\usepackage{colortbl}
\usepackage{bm}
\usepackage[nospace]{cite}

\lstset{basicstyle=\scriptsize,breaklines=true}

%\linespread{0.94}%

\makeatletter
\def\@copyrightspace{\relax}
\makeatother

\begin{document}

\setlength{\belowdisplayskip}{2pt} \setlength{\belowdisplayshortskip}{2pt}
\setlength{\abovedisplayskip}{2pt} \setlength{\abovedisplayshortskip}{2pt}
\setlength{\belowcaptionskip}{-8pt}
\selectfont

\newtheorem{theorem}{Theorem}
\newtheorem{example}{Example}
\newtheorem{definition}{Definition}
\newtheorem{proposition}{Proposition}
\newtheorem{lemma}{Lemma}
\newtheorem{corollary}{Corollary}

\newcommand{\cond}{\textrm{Cond}\xspace}
\newcommand{\dataset}{data set\xspace}
\newcommand{\datasets}{data sets\xspace}
\newcommand{\spview}{\textsf{SPView}\xspace}
\newcommand{\fjview}{\textsf{FJView}\xspace}
\newcommand{\aggview}{\textsf{AggView}\xspace}
\newcommand{\hashfunc}[1]{\textsf{hash}(#1)\xspace}
\newcommand{\hashop}{\textsf{hash}\xspace}
\newcommand{\nsc}{\textsf{NormalizedSC}\xspace}
\newcommand{\rsc}{\textsf{RawSC}\xspace}

\newcommand{\avgfunc}{\ensuremath{\texttt{avg} }\xspace}
\newcommand{\maxfunc}{\ensuremath{\texttt{max} }\xspace}
\newcommand{\minfunc}{\ensuremath{\texttt{min} }\xspace}
\newcommand{\histfunc}{\ensuremath{\texttt{histogram\_numeric} }\xspace}
\newcommand{\countfunc}{\ensuremath{\texttt{count}}\xspace}
\newcommand{\sumfunc}{\ensuremath{\texttt{sum} }\xspace}
\newcommand{\varfunc}{\ensuremath{\texttt{var} }\xspace}
\newcommand{\covfunc}{\ensuremath{\texttt{cov} }\xspace}
\newcommand{\corrfunc}{\ensuremath{\texttt{corr} }\xspace}
\newcommand{\medfunc}{\ensuremath{\texttt{median} }\xspace}
\newcommand{\percfunc}{\ensuremath{\texttt{percentile} }\xspace}
\newcommand{\havingfunc}{\ensuremath{\texttt{HAVING} }\xspace}
\newcommand{\selectfunc}{\ensuremath{\texttt{select} }\xspace}
\newcommand{\ratio}{\ensuremath{\rho }\xspace}


\newcommand{\insertion}{\ensuremath{\texttt{INSERT} }\xspace}
\newcommand{\update}{\ensuremath{\texttt{UPDATE} }\xspace}
\newcommand{\delete}{\ensuremath{\texttt{DELETE} }\xspace}


\newcommand{\tbl}[1]{\textsf{#1}\xspace}
\newcommand{\field}[1]{\textsf{#1}\xspace}
\newcommand{\cost}{\textrm{cost}\xspace}
\newcommand{\ans}{\textsf{ans}\xspace}
\newcommand{\dans}{\Delta\textsf{ans}\xspace}
\newcommand{\cqp}{correction query processing\xspace}
\newcommand{\Cqp}{Correction query processing\xspace}

\newcommand{\reminder}[1]{{{\textcolor{magenta}{\{\{\bf #1\}\}}}\xspace}}
\newcommand{\specialcell}[2][c]{%
  \begin{tabular}[#1]{@{}c@{}}#2\end{tabular}}

\def\ojoin{\setbox0=\hbox{$\bowtie$}%
  \rule[-.02ex]{.25em}{.4pt}\llap{\rule[\ht0]{.25em}{.4pt}}}
\def\leftouterjoin{\mathbin{\ojoin\mkern-5.8mu\bowtie}}
\def\rightouterjoin{\mathbin{\bowtie\mkern-5.8mu\ojoin}}
\def\fullouterjoin{\mathbin{\ojoin\mkern-5.8mu\bowtie\mkern-5.8mu\ojoin}}

%\newcommand{\reminder}[1] {}
\pagestyle{plain}

\title{Sample-View-Clean: A Data Cleaning Approach for \\Fresh Query Answers From Stale Materialized Views}

\maketitle
\iffalse

\begin{abstract}
Materialized views, stored pre-computed query results, are used to facilitate queries on large datasets.
Materialized views can become out-date when base tables change and queries will give \emph{stale} results.
Maintaining materialized views has been well studied but is often very costly.
In this work, we model the view maintenance problem as a data cleaning problem and treat out-of-date rows materialized views as a type of data error.
In particular, we analyze the maintenance procedure for the views and infer how these updates affect a queries on a the stale view.
For common aggregate queries (\sumfunc, \countfunc, and \avgfunc), we can derive a correction factor from the sample to ``clean" the stale query result.
We bound our corrections in analytic confidence intervals, and in fact, we show that our technique is optimal with respect to estimate variance.
As sampling can be sensitive to long-tailed distributions, we further consider an outlier indexing technique to give increased accuracy when the data distributions are skewed.
We evaluate this approach on real and synthetic datasets and in both a single node (MySQL) and distributed environment (Apache Spark).
In one large scale experiment using 20-node Apache Spark cluster, we derived a 700,000,000 row view from a 1TB activity log base dataset.
We found that maintaining a 5\% sample was 16.3x faster than full maintenance yet only with mean query error of 1.2\%.
\end{abstract}
\fi

\vspace{-1em}

\begin{abstract}
Materialized views, stored pre-computed query results, are widely used to facilitate fast queries on large datasets. 
In the Big Data era when new data arrives at an increasingly fast rate, maintaining materialized views can be costly.
Due to this cost, maintenance is often deferred to a later time, but at the new cost of returning inaccurate query results on stale views.
In this paper, we address this problem from a data cleaning perspective and model staleness as a type of data error.
We take inspiration from recent data cleaning results that greatly improve query accuracy by cleaning only a small sample of dirty data.
To complement existing deferred maintenance approaches, we maintain a sample of a materialized view to correct stale query results between maintenance periods, without the cost of full maintenance. 
For common aggregate queries (SUM, COUNT, and AVG), we bound our corrections in analytic confidence intervals, and prove that our technique is optimal with respect to estimate variance. 
As sampling can be sensitive to long-tailed distributions, we further explore an outlier indexing technique to give increased accuracy when the data distributions are skewed. 
We evaluate our method on real and synthetic datasets and results suggest that we are able to provide more accurate query results without having to maintain the full view. 
\end{abstract}


\vspace{-0.5em}
\section{Introduction}
Materialized views (MVs), storing pre-computed query results, are a well-studied approach to speed up queries on large datasets \cite{LarsonY85, gupta1995maintenance, chirkova2011materialized, halevy2001answering}.
During the last 30 years, the research community has thoroughly studied MVs, and all major database vendors have added support for them.
In a world of ever-increasing data sizes, MVs are becoming even more important, both for traditional query processing
 \cite{lefevre2014opportunistic, bailis2014scalable, perez2014history} and for more advanced analytics based on linear algebra and machine learning \cite{nikolic2014linview, zhang2014mat}.

However, when the underlying data is changed MVs can become \emph{stale} and all queries that use a stale MV can give incorrect answers. 
One solution would be to recompute the MV every time a change occurs, however, in many cases, it is more efficent to incrementally update the MV instead of recomputing the query.
There has been substantial work in deriving these incremental updates (incremental maintenance algorithms) for different classes of MVs and optimizing their execution \cite{gupta1995maintenance, DBLP:conf/sigmod/GriffinL95, griffin1997improved, samtani1999self, DBLP:conf/sac/TrutaC07, DBLP:journals/vldb/KochAKNNLS14, chirkova2011materialized}.

For frequently changing tables even incremental maintenance can be expensive; every update to the underlying data requires updating all the dependent views.  
This problem is exacerbated in Big Data environments, where new records arrive at an increasingly fast rate and where data are often 
distributed across multiple machines.  
As a result, in production environments it is common to defer view maintenance to a later time \cite{chirkova2011materialized, zhou2007lazy, DBLP:conf/sigmod/ColbyGLMT96} so that updates can be batched together to amortize overheads and maintenance work can be scheduled for times of low system utilization.  

%While deferring maintenance has compelling benefits, it unfortunately brings its own costs, as views become increasingly stale in-between maintenance periods.   
%The problem of stale MVs parallels the problem of dirty data studied in data cleaning \cite{rahm2000data}; as both staleness and erroneous records lead to inaccurate query answers.
%The observation that stale MVs are a type of dirty data leads us to the key insight behind our work; namely, that data cleaning techniques can be used to mitigate the negative impacts of deferred MV maintenance.  

While deferring maintenance has compelling benefits, it unfortunately brings its own costs, namely that views become increasingly stale in-between maintenance periods. As a result, queries on those views can return increasingly incorrect answers.
The problem of stale MVs parallels the problem of dirty data studied in data cleaning~\cite{rahm2000data}; as both staleness and erroneous records lead to inaccurate query answers.
The observation that stale MVs are a type of dirty data leads us to the key insight behind our work; namely, that data cleaning techniques can be used to mitigate the negative impacts
of deferred MV maintenance.

\begin{figure}[t] \vspace{-2em}
\centering
 \includegraphics[scale=0.30]{figs/sys-arch.png} \vspace{-.25em}
 \caption{If a view is being periodically maintained, in the interim, query results may be stale. Sample-View-Clean sits above an existing view maintenance architecture, and provides an interface for approximately up-to-date query results by maintaining a sample. The user can tune the sampling ratio to meet the resource constraints of the system \reminder{You never mention outlier index in the Introduction except for the contributions.}\label{sys-arch}}\vspace{-1.75em}
\end{figure}

Data cleaning has been studied extensively in the literature (e.g., see Rahm and Do for a survey\cite{rahm2000data}) but increasing data volumes and arrival rates have led to development of new, efficient sampling-based approaches for coping with dirty data.   
In our prior work, we developed the SampleClean framework to greatly improve query accuracy while cleaning only a small sample of dirty records \cite{wang1999sample}.  
We proposed an approach called \nsc that corrects dirty query results by using a sample of clean data to learn how the dirtiness affects that query and then calculates a correction.  
This perspective raises a new possibility for MVs: we can use a sample of clean (up-to-date) data to return more accurate query results without incurring the cost of full view maintenance.
Of course, the metaphor of stale MVs as dirty data only goes so far. 
View staleness is a different type of error than typical dirty data, which raises interesting new challenges in sampling, cleaning, and efficient query processing.

To address these new challenges, we propose Sample-View-Clean (SVC), a framework that applies sampling to approximately correct query results on stale views.
SVC takes a sample of up-to-date rows from the view, and extrapolates a correction factor for query answers on the stale view. Figure~\ref{sys-arch} shows how SVC can be used as complementary to existing deferred maintenance approaches. When the MVs become stale between maintenance cycles, we apply SVC for query result estimation for a far smaller cost than having to maintain the entire view.
The query results from SVC are in expectation up-to-date. While they are approximate results, the approximate error is more manageable than staleness: (1) the uniformity of sampling allows us to apply theory from statistics such as the Central Limit Theorem to give tight bounds on approximate results, and (2) the approximate error is parameterized by the sample size as opposed to potentially unbounded staleness.

%The theoretical scope of the SVC is quite general, and our approach can be applied, in principle, to any multiset-valued materialized view.
In practice, both MVs and queries can be arbitrarily complex.
Since SVC is based on sampling, there are a subclass of views for which SVC can save significant computation and a subclass of queries on these views for which SVC can give accurate corrections.
In this work, we explore these classes from both a theoretical perspective (i.e. when is our query correction optimal w.r.t estimate variance) and an empirical perspective for queries that do not satisfy the optimality conditions when is SVC still beneficial.



%Instead of a binary model of a view being either fully up-to-date (and all queries on it are correct) or stale (and all queries are possibly incorrect), SVC uses sampling to give a tradeoff between the computation required for maintenance and query accuracy.



%The key technical insight is that instead of randomly sampling the updates to the base relations as is common in the streaming literature \cite{babcock2002sampling, DBLP:journals/pvldb/MankuM12}, we apply a deterministic \textbf{hash} operation to sample the rows in a view.
%Then, working backwards through relational expression of the view using lineage \cite{DBLP:journals/vldb/CuiW03}, we do just enough work to maintain those rows.
%The result is an up-to-date uniform sample which can be used to give unbiased corrections of aggregate queries (e.g. \sumfunc, \countfunc, \avgfunc), and in fact, we can support the same set of aggregate queries supported in the AQP literature \cite{AgarwalMPMMS13, agarwalknowing}.
%Since sampling is known to be sensitive to outliers (i.e., rows that have abnormal attribute values), we
%utilize a technique called outlier indexing \cite{chaudhuri2001overcoming} to ensure that MV rows that are derived from ``outlier" records are accounted for in the sample which we found significantly increases accuracy in skewed data sets.



To summarize, our contributions are as follows:
\begin{itemize}\vspace{-.45em}
\item We model the incremental maintenance problem as a data cleaning problem and staleness as a type of data error.\vspace{-.45em}
\item We show how sampling with a \hashop operation can reduce computation during maintenance. \vspace{-.45em}
\item We derive a correction for aggregate queries using the sample and show that, in fact, our correction is optimal for \sumfunc, \countfunc, and \avgfunc. \vspace{-.45em}
\item We use an outlier index to increase the accuracy of the approach for power-law, long-tailed, and skewed distributions.\vspace{-.45em}
\item We evaluate our approach on a single-node MySQL database with a 10GB skewed TPCD benchmark dataset and on a 20-node Apache Spark cluster with a 1TB log dataset from a video streaming company. Our results show that sampling is significantly faster than full view maintenance, and also can give highly accurate results for a variety of queries and views.\vspace{-.45em}
\end{itemize}

The paper is organized as follows: 
In Section~\ref{sec-background}, we give the necessary background for our work.
Next, in Section~\ref{sec-arch}, we formalize the problem.
In Section~\ref{sampling} and~\ref{correction}, we describe the sampling and query processing of our technique.
In Section~\ref{outlier}, we describe the outlier indexing framework.
In Section~\ref{sec:ext}, we discuss extensions to our framework.
Then, in Section~\ref{exp}, we evaluate our approach.
Finally, we discuss Related Work in Section~\ref{related} and present our Conclusions and Future Work in Section~\ref{conclusion}.

\section{Background}\label{sec-background}
In this section, we briefly overview the current challenges with view maintenance and
our prior work in scalable data cleaning.

%\begin{figure}[t] 
%\centering
%\vspace{-0.75em}
% \includegraphics[width=\columnwidth]{figs/sample-clean-example.png}\vspace{-0.25em}
% \caption{A simplified log analysis example dataset. In this dataset, there are two tables: a fact table representing video views and a dimension table representing the videos.\label{example-1}}\vspace{-1em}
%\end{figure}

\subsection{Running Example: Log Analysis}
To illustrate our framework, we use the following running example which is a 
simplified schema of one of our experimental datasets.%(Figure~\ref{example-1}).
Imagine, we are querying logs from a video streaming company. 
These logs record visits from users as they happen and grow over time.
We have two tables, \tbl{Log} and \tbl{Video}, with the following schema:

\begin{lstlisting}[mathescape]
Log(sessionId$\textrm{,}$ videoId$\textrm{,}$ responseTime$\textrm{,}$ userAgent)
Video(videoId$\textrm{,}$ title$\textrm{,}$ duration)
\end{lstlisting}
These tables are related with a foreign-key relationship between
Log and Video.
Though SVC supports inserts, deletions, and updates, for clarity in our example, we consider insertions
into Log which is cached in a temporary table:

%\reminder{I replace LogInserts with LogIns for saving a line of space. Make sure it is consistent in the whole paper.}
\begin{lstlisting}[mathescape]
LogIns(sessionId$\textrm{,}$ videoId$\textrm{,}$ responseTime$\textrm{,}$ userAgent)
\end{lstlisting}


\iffalse
\subsection{Materialized View Maintenance}\label{subsec-inc}
Views define logical relations which can be queried instead of physical base relations.
MVs are a class of views that are pre-computed and stored (i.e materialized).
Any form of pre-computed, derived data encounters the problem of staleness when the physical base relations update.

One approach to this problem is to recompute the materialized view every time there are updates to the base tables.
However, this approach is very inefficient if updates to the data generally have small or sparse effect on the MV. 
A contrasting approach is incremental view maintenance (IVM), where rows in the MV are incrementally updated based on the updates to the base table.
Incremental maintenance of MVs has been well studied; see \cite{chirkova2011materialized} for a survey of the approaches. 
At a high-level, incremental maintenance algorithms typically consist of the following steps: (1) maintain a cache of insertions and deletions for each physical base table, then using the view definition derive a \emph{change propagation formula} in terms of the set of insertions and deletions, and finally apply the formula to the view.
For a variety of view types, these rules are described in detail in \cite{DBLP:journals/vldb/KochAKNNLS14, DBLP:conf/pods/Koch10}.

%Incremental maintenance may not be efficient in all cases.
%Consider the view that calculates the median \tbl{responseTime} grouped by \tbl{userAgent} on our running example dataset.
%In general, to ensure correctness, the view has to store the entire set of \tbl{responseTime} attributes for each group to allow for incremental maintenance.
%Along the lines of this example, there are cases when recomputation may require less storage of state or even less computation.
%Thus, materialized views are maintained either with incremental maintenance, recomputation, or a mix.

In real-world systems, for large datasets or fast data update rate, it may not always be feasible to maintain MVs immediately. 
Therefore, deferring maintenance (periodically or adaptively) is an alternative and often preferred solution.
The main insight of deferral is to avoid maintaining the view immediately and to schedule an update at a more convenient time.
In deferred maintenance approaches, the user often accepts some degree of staleness for additional flexibility in scheduling.
%SVC offers a data cleaning perspective on this problem, namely, there is a trade-off between the query result accuracy and computation.
By using sampling, we give the user access to a new trade-off space between immediate (or close to immediate, i.e., mini-batch) maintenance and long-periodic maintenance.

%In particular, we highlight a technique called lazy maintenance which applies updates to the view only when a user's query requires a row \cite{zhou2007lazy}.
%While always fresh, both lazy maintenance and immediate maintenance hit a bottleneck when there are rapid updates, and this results increasingly degraded performance if a user wants to query a view.
%The alternative is a periodic strategy, but this means that there is unbounded error on queries between maintenance periods.
%\subsubsection{Practical Considerations}
\fi


\subsection{SampleClean: Fast and Accurate Query Processing on Dirty Data}
In our prior work on the SampleClean project \cite{wang1999sample}, we proposed a framework for scalable data cleaning.
Similar to the accuracy-performance contrast between immediate maintenance and periodic maintenance in the materialized view setting, data cleaning also faces a similar challenge.
Traditionally, data cleaning has explored expensive, up-front cleaning of entire datasets for increased query accuracy, and those who were unwilling to pay the full cleaning cost avoided data cleaning altogether.
We proposed SampleClean to add an additional trade-off to this design space by using sampling.

SampleClean has three parts: (1) sampling, (2) data cleaning, and (3) query result estimation.
First, SampleClean creates a sample of dirty data (which are erroneous, missing, or otherwise corrupted records).
Then, the framework applies a data cleaning procedure to the sample.
Finally, when users query the dataset, the framework uses the clean sample to extrapolate clean query results.
In this work, the main challenge was that data cleaning can potentially change the statistics of a sample and the queries need to compensate for those effects.
In our initial work, SampleClean mainly focused on three common aggregates: \sumfunc, \avgfunc, and \countfunc queries.

The SampleClean project showed that there were two contrasting approaches to query processing on a sample of cleaned data.
We could (1) clean the sample first and then run the query on the sample, or (2) look at the difference between the clean and dirty samples and calculate a correction to correct an existing dirty result. 
Approach (1) is similar to those studied in the Approximate Query Processing (AQP) literature \cite{OlkenR86,AgarwalMPMMS13, joshi2008materialized}; however in cases when data errors were small, we found that the approximation error dominated.
To address this problem, we developed approach (2), which we called \nsc, and we found that \nsc was more accurate in mostly clean datasets as it leverages existing deterministic results leading to reduced approximation error.
In the materialized view setting, we found that \nsc led to more accurate results in our experiments (see Section \ref{exp}). 

%\begin{figure}[t] \vspace{-2em}
%\centering
% \includegraphics[scale=0.26]{figs/sys-arch2.pdf} \vspace{-.25em}
% \caption{A basic overview of SampleClean. SampleClean uses a random sample of dirty data to learn how a data cleaning algorithm affects queries on the sample. We can then derive a correction to compensate for the dirtiness.\label{sc}}\vspace{-1.75em}
%\end{figure}

\subsection{New Challenges}
%Inspired by SampleClean, SVC samples a stale view, cleans the sample view by restricting the maintenance to just the rows in the sample, and then applies \nsc to correct the results of queries on the stale view.
Applying this data cleaning framework to the materialized view setting leads to some interesting theoretical challenges with new insights for both materialized view maintenance and data cleaning.
%In SampleClean, we simply treated the data cleaning as a black box and did not study how to clean a dirty sample data. 
%However, in SVC, we cannot make such assumption and have to devise efficient data-cleaning techniques to ``clean" a stale sample view. 
Staleness is a new type of data error.
In materialized views, staleness can lead to rows that are missing from the ``dirty" view or conversely need to be deleted. These issues pose new challenges in  query correction. 
We further explore and formalize the class of queries that SVC can support.
We extend the generality of the framework to support queries than the \sumfunc, \avgfunc, and \countfunc which studied before.
Sampling is particularly sensitive to variance in the dataset, and large outliers can significantly reduce query accuracy.
In this work, we give an explicit treatment of outliers records. 





\section{System Overview}
Frequent incremental maintenance of materialized views can be costly or infeasible given system constraints such as throughput. 
We propose approximate query correction to address this problem, and to give a flexible tradeoff between query accuracy and system performance.
The key idea is that given a stale aggregation query (SUM, COUNT, AVG, VAR) on a materialized view, we use a sample of up-to-date data to estimate how much that data changes the stale aggregation query result.
From the sample, we can derive a correction to the stale query result, which in expectation is accuracte, but has variance inherrent to sampling.

Our proposed system will work in conjunction with existing maintenance or re-calculation approaches.
We envision the scenario where materialized views are being refreshed periodically eg. nightly.
While maintaining the entire view throughout the day may be infeasible, sampling allows the database to scale maintenance with the performance and resource constraints during the day.
Then, between maintenance periods, we can provide approximately up-to-date query results for aggregation queries.

However, a challenge is that supporting only aggregation queries has the potential mask outliers in the updates.
Since, we can answer selection queries on the view, we may miss outlier rows either because they were not sampled or because
they are averaged into an aggregation query.
We define an ``outlier index" on the base table which indexes records with abnormal attribute values, where abnormality is defined by user specified rules.
Therefore, we couple our sampling with outlier indexing to guarantee that rows in the materialized views derived outlier base table records are in the sample.
The result is that we can answer selection queries exactly on these outlier rows which are often the most queried rows in the materialized view.
What is particularly surprising is that outliers can give valuable information about the data distribution and these can be used to improve the accuracy of our aggregation queries as well.

To illustrate our approach, we use the following running example which is a 
simplified schema of one of our experimental datasets (Figure \ref{example}).
Imagine, we are querying logs from a video streaming company. 
These logs record visits from users as they happen and grow over time.
We have two table: Log and Video, with the following schema:
\begin{lstlisting}
Log(sessionID, videoID, responseTime, userAgent)
Video(videoID, title, duration)
\end{lstlisting}
These tables are related with a foreign-key relationship between
Log and Video, and there is an integrity constraint that every log
record must link to one video in the video table.

\begin{figure}[h]
\label{example}
\centering
 \includegraphics[width=\columnwidth]{figs/sample-clean-example.png}
 \caption{TODO}
\end{figure}

\subsection{System Architecture}
The architecture of our proposed solution is shown in Figure \ref{sys-arch}.
The left side of the diagram resembles a tradition view maintenance architecture.
However, there are a couple of key additions: (1) instead of maintaining the view,
we issue corrections to query results on the view, (2) to acheive this we maintain
an up-to-date sample, and (3) we have an outlier index.
In this section, we will overview these three components and contrast our approach 
to other materialized view and AQP architectures.

\begin{figure}[h]
\label{sys-arch}
\centering
 \includegraphics[width=\columnwidth]{figs/sys-arch.png}
 \caption{TODO}
\end{figure}

\subsubsection{Supported Materialized Views}\label{subsubsec:supported-view}
We will first introduce the taxonomy of materialized views
that can benefit from our approach. In particular, we provide example
situations when view maintenance can be costly. 

\vspace{1em}

\noindent\textbf{Select-Project Views}

One type of view that we consider are views generated from Select-Project
expressions of the following form:

\begin{lstlisting}
SELECT [col1,col2,...] 
FROM table 
WHERE condition([col1,col2,...]) 
\end{lstlisting}

There are situations when such views are expensive to maintain. For
example, as often the case with activity logs, the base table may
contain semi-structured data that requires parsing or preprocessing
as a part of the view definition. Consider the following example,
a typical column in online activity logs is the User-Agent String
(Figure ?). When a user accesses a webpage the browser reports this
string to identify the browser type, operating system, and layout
engine. Suppose, we wanted to create the following view:

\begin{lstlisting}
SELECT * FROM Log 
WHERE userAgent 
LIKE '%Mozilla%'
\end{lstlisting}

This involves evaluating regular expression on the string to see if
it matches a criteria.
Testing a complex regular expression will be far more expensive than
numerical comparisions or equality testing. In more extreme examples,
the columns may be serialized objects (eg. represented in JSON) which
need to be deserialized before evaluating a predicate.

%\begin{figure}
%\includegraphics[scale=0.3]{Documents/Research/sigmod15/Images/user-agent}

%\caption{User-Agent Activity Logs. <TODO>}
%\end{figure}

\vspace{1em}

\noindent\textbf{Aggregation Views}

We also consider aggregation views of the following form:

\begin{lstlisting}
SELECT [f1(col1),f2(col2),...] 
FROM table 
WHERE condition([col1,col2,...]) 
GROUP BY [col1,col2,...]
\end{lstlisting}

While the same costs that Select-Project views can incur due to pre-process
apply as well, aggregation views pose additional challenges to incremental
view maintenance. Aggregation views can be costly to maintain when
the cardinality of the result is large; that is when there group by
clause is very selective. Consider the following example:

\begin{lstlisting}
SELECT videoID, 
max(responseTime) AS maxResponseTime 
FROM Log 
GROUP BY videoID;
\end{lstlisting}

The cardinality of the delta view is the total number of videos in the log which can be very large.
Thus it will be costly to propagate this result with the existing view, potentially updating a large number of rows. 
These costs increase in a distributed environment where a larger
delta view means that more data has to be communicated through a shuffle
operation. 

\vspace{1em}

\noindent\textbf{Foreign-Key Join Views}

The third type of view we consider are Foreign-Key Join views:

\begin{lstlisting}
SELECT table1.[col1,col2,...], 
table2.[col1,col2,...]
FROM table1, table2 
WHERE table1.fk = table2.fk 
AND condition([col1,col2,...]) 
\end{lstlisting}

Such views are ubquitious in star schemas {[}?{]} and can be particularly
costly to maintain in distributed environments. Consider the following example:

\begin{lstlisting}
SELECT * 
FROM Log, Video 
WHERE Log.videoID = Video.videoID;
\end{lstlisting}

Suppose new recrods have been inserted into Log. Calculating the
delta view involves joining the new records with the entire table Video.
While indexing is the prefered strategy to optimize such joins, many
distributed systems, such as Apache Spark, Cloudera Impala, and Apache
Tez, lack native support for join indices. To avoid scanning the entire
table, these systems rely on partitioned joins where records linked
by foreign keys are stored on the same partition. However, when these
join keys cross partition lines this operation can become increasingly
expensive.

\subsubsection{Up-to-date Samples}
We presented examples where these views can be expensive to maintain.
In this work, we address the question of whether we need to maintain
the entire view to answer aggregate queries on these views. Our proposed
solution is to sample the delta view $\Delta\textbf{V}$, and incrementally
maintain just a sample of $\textbf{V}_{T}$. For example, in our Select-Project
view example application, we would have to parse only a sample of
the inserted records. Similarly, for the Aggregation view, sampling
$\Delta\textbf{V}$ reduces the cardinality of the result and consequently
communication/merging costs. And finally, for the Join views, we would
only have to join a sample of the inserted records. 

\subsubsection{Outlier Indexing}
We are often interested in records that outliers, 
which we define in this work as records with abnormally large attribute values.
Outliers and power-law distributions are a common property in web-scale datasets.
Often the queries of interest involve the outlier records, however sampling does 
have the potential to mask outliers in the updates.
If we have a small sampling ratio, more likely than not, outliers will be missed.

Therefore, we propose coupling sampling with outlier indexing. 
That is, we guarantee that records (or rows in the view derived from those records) 
with abnormally large attribute values are included in the sample.
What is particularly interesting is that these records give information about the distribution 
and can be used to reduce variance in our estimates.

\subsubsection{Alternative Architectures}
\noindent\textbf{No Maintenance: }
One approach for up-to-date results is to avoid incremental maintenance altogether.
In this approach, one would periodically re-calculate the views.
This approach allows for the highest throughput in terms of records written, but can
suffer from long periods of stale data.

\vspace{1em}

\noindent\textbf{View Maintenance: }
The other option is to fully maintain the materialized view. 
While immediate maintenance offers consistent results, 
it is at a steep computational cost.
So often incremental maintenance, is deferred or done periodically [?].
These deferred schemes can lead to period of staleness as well.

\vspace{1em}

\noindent\textbf{SAQP: }
Estimating the results of aggregate queries from samples has been
well studied in a field called Sample-based Approximate Query Processing
(SAQP). While the concept of estimating a correction from a sample
is similar to SAQP it differs in a few critical ways. Traditional
SAQP techniques apply their sampling directly to base tables and not
on views. The SAQP approach to this problem, would be to treat aggregate
queries on views as nested queries and then apply them to a sample
of the base data {[}?{]}. Another potential technique would be to
estimate the result directly from the maintained sample; a sort of
SAQP scheme on the sample of the view. We found that empricially estimating
a correction and leveraging an existing deterministic result lead
to lower variance results on real datasets (see Section ?). We analyze
the tradeoffs of these techniques in the following sections.

\vspace{1em}

Our approach provides a flexible tradeoff between the performance benefits of No Maintenance and the Consistency benefits of full maintenance. Accordingly, we evaluate our approach against the accuracy of not maintaining the data and the performance of maintaining the data.
We further compare our approach to SAQP, configured so it would have the same cost, in terms of accuracy.
\section{Efficient Maintenance With Sampling} \label{sampling}
%\reminder{Make sure that we do a good survey on ``sampling from a view" and discuss them in the related work, e.g., [Frank et al., VLDB 86], [Nirkhiwale et al., VLDB 13]}
In the previous section, we formalized the computation of $\hat{S'}$  as a data cleaning procedure.
A naive solution to derive a sample $\hat{S'}$ is to just apply the maintenance strategy and then sample.
However, this does not make the maintenance of the sample any more efficient.
Ideally, we want to integrate the sampling into the maintenance strategy $\mathcal{M}$ so that expensive operators
need not operate on the full data.
In this section, we discuss how to efficiently derive $\hat{S'}$ and the conditions under which
maintaining $\hat{S'}$ is much cheaper than maintaining the entire view $S'$.

\subsection{Uniform Sampling on Views}
%In this work, we focus on result estimation on uniform samples of views. \reminder{You have already explained uniform sampling in Sec 3.2.}
For a sampling ratio $m$, we call a sample view $\hat{S'}$ a uniform sample of $S'$, under the following condition:

\begin{definition}[Uniform Sample] We say the relation $\hat{S'}$ is a \emph{uniform sample} of $S'$ if
\begin{enumerate}[label=(\arabic*),itemsep=1pt]
\item $\forall s \in \hat{S'} : s \in S'$;
\item $Pr(s_1 \in \hat{S'}) =  Pr(s_2 \in \hat{S'}) = m$.
\end{enumerate}
\end{definition}

A traditional ``coin-flip" sampling algorithm is not suited for this property as it is known that such sampling commutes very poorly many relational operations such as joins and aggregates \cite{chaudhuri1999random}.
Recall, the view in our example \textsf{countView}. 
If we sampled the base relation the result would have a mix of missing rows from the view and rows with incorrect aggregates.
However, this is not what we require.
We want a uniform sample of the rows in the view. 

To get a uniform sample of a view, the main problem is that for every row sampled in the view, our sampling technique needs to include all of the rows that contribute to its materialization.
Achieving this requires a definition of lineage; traceable, unique identification for rows.

\subsection{Identification With Row Lineage}
\label{lin}
Lineage has been an important tool in the analysis of materialized views \cite{DBLP:journals/vldb/CuiW03} and in approximate query processing \cite{DBLP:conf/sigmod/ZengGMZ14}. %\reminder{Add Kai into acknownledgement for helping us with problem formulation. }
We recursively define a set of consistent primary keys for all nodes in the expression tree:
\begin{definition} [Primary Key]
For every relational expression $R$, we define the primary key of every subexpression to be:
\begin{itemize}[noitemsep,leftmargin=*]
\item Base Case: All relations (leaves) must have an attribute $p$ which is designated as a primary key. That uniquely identifies rows.
\item $\sigma_{\phi}(R)$: Primary key of the result is the primary key of R 
\item $\Pi_{(a_1,...,a_k)}(R)$: Primary key of the result is the primary key of R. The primary key must always be included in the projection.
\item $\bowtie_{\phi (r1,r2)}(R_1,R_2)$: The primary key of the result is the union of the primary key of $R_1$ and $R_2$. 
\item $\gamma_{f,A}(R)$: The primary key of the result is the tuple $A$.
\item $R_1 \cup R_2$: Primary key of the result is the primary key of~R
\item $R_1 \cap R_2$: Primary key of the result is the primary key of~R
\item $R_1 - R_2$: Primary key of the result is the primary key of~R
\end{itemize}
\end{definition}
This definition of a primary key for a relational expression, allows us to trace the primary key through the expression tree.
Our definition of the primary key is also constructive; that is, if an expression has a null primary key then we modify every projection operation to ensure that primary key of the subrelation is never projected out.

\subsection{Hashing Operator}
\label{push}
If we have a deterministic way of mapping a primary key defined in the previous subsection to a sample we can also ensure that all contributing expressions are also sampled. 
To achieve this we use a hashing procedure.
Let us denote the hashing operator $\eta_{a, m}(R)$. 
For all tuples in R, this operator applies a hash function whose range is $[0,1]$ to primary key $a$ (which may be a set) and selects those records with hash value less than or equal to $m$.
If the hash function is sufficiently uniform, then $h(a) \le m$ samples on average a fraction $m$ of the tuples.
%This definition is without loss of generality for uniform hash function, as if we have a hash function whose range is the set of integers (as implemented in MySQL or Apache Hive) we can take the absolute value and divide by the maximum integer mapping this range back $[0,1]$. 

To achieve the performance benefits of sampling, we push down the hashing operator through the query tree.
The further than we can push $\eta$ down the query tree, the more operators can benefit from the sampling.
However, it is important to note that for some of the expressions, notably joins, the push down rules are more complex. 
It turns out in general we cannot push down even a deterministic sample through those expressions.
We formalize the push down rules below:
\begin{definition}[Hash Pushdown]
Let $a$ be a primary key of a materialized view. The following rules can be applied to push $\eta_{a, m}(R)$ down the expression tree of the maintenance strategy. 
\begin{itemize}[noitemsep]
\item $\sigma_{\phi}(R)$: Push $\eta$ through the expression.  
\item $\Pi_{p,[a_2,...,a_k]}(R)$: Push $\eta $ through if $a$ is in the projection.
\item $\bowtie_{\phi (r1,r2)}(R_1,R_2)$: Blocks $\eta $ in general. 
\item $\gamma_{f,A}(R)$: Push $\eta $ through if $a$ is in the group by clause $A$.
\item $R_1 \cup R_2$: Push $\eta $ through to both $R_1$ and $R_2$
\item $R_1 \cap R_2$: Push $\eta $ through to both $R_1$ and $R_2$
\item $R_1 - R_2$: Push $\eta $ through to both $R_1$ and $R_2$
\end{itemize}
\end{definition}
In special cases, we can push the hashing operator down through joins. 

\textbf{Many-to-one Join: } If we have an join with two relations $R_1$ and $R_2$ and we know that for every $r_1 \in R_1$ there is at most one $r_2$ in $R_2$ that satisfies the join condition, then we push $\eta$ down to $R_1$.

\textbf{One-to-one Join: } If we have the previous condition and also the converse is true for every $r_2 \in R_2$ there is at most one $r_1$ in $R_1$, then we push $\eta$ down to $R_1$ and $R_2$.

\textbf{(Semi/Anti)-Join: } We can always push $\eta$ down on Semi-joins. For anti-joins we can push $\eta$ down because we can rewrite the node as $R_1 \dot{-} (R_1 \ltimes R_2) $ and apply the pushdown rules for set difference and Semi-Joins.

\subsection{Hashing and Correspondence}
Another benefit of deterministic hashing is that when applied in conjunction to the primary keys of a view, we get the Correspondence Property (Definition \ref{correspondence}) for free.
\begin{proposition}[Hashing Correspondence]
Suppose we have $S$ which is the stale view and $S'$ which is the up-to-date view.
Both these views have the same schema and a primary key $a$.
Let $\eta_{a, m}$ be our hash function that applies the hashing to the primary key $a$.
\[
\hat{S} = \eta_{a, m}(S)
\]
\[
\hat{S'} = \eta_{a, m}(S')
\]
Then, two samples $\hat{S'}$ and $\hat{S}$ correspond.
\end{proposition}
\begin{proof}[Sketch]
Since the primary keys are key consistent between $\hat{S'}$ and $\hat{S}$, included and excluded rows are preserved by the hashing.
See the extended version for a full proof \reminder{TR}.
\iffalse
There are four conditions for correspondence:
\begin{itemize}[noitemsep]
\item 1. For every row $r$ in $\hat{S}$ that required a delete, $r \not\in \hat{S'}$
\item 2. For every row $r$ in $\hat{S}$ that required an update, $r\in \hat{S'}$
\item 3. For every row $r$ in $\hat{S}$  that was unchanged, $r \in \hat{S'}$
\item 4. For every row $r$ in $S$ but not in $\hat{S}$, $r \not\in \hat{S'}$
\end{itemize}
Condition 1 is satisfied since if $r$ is deleted, then $r \not \in S'$ which implies that $r \not\in \hat{S'}$.
Condition 2 and 3 are satisfied since if $r$ is in $\hat{S}$ then it was sampled, and then since the primary key is consistent between $S$ and $S'$ it will also be sampled in $\hat{S'}$.
Condition 4 is just the converse of 2 and 3 so it is satisfied.
\fi
\end{proof}
We will use this property in the next section to get estimates for queries on the materialized view.

\subsection{Example}
We will illustrate our proposed approach on our example view \textsf{countView}.
The maintenance strategy of this view is described in the previous section.
Based on the rules described in Section \ref{lin}, the primary key of the view is \textsf{videoId}.
We can apply our sampling operator to this key, and use the pushdown rules described in Section \ref{push} to efficiently sample the maintenance strategy.

In Figure \ref{exexpr2}, we illustrate the pushdown process.
The the first operator we see in the expression tree is a projection that increments the \textsf{visitCount} in the view, and this allows
for push down since \textsf{videoId} is in the projection.
The second expression is a one-to-one join which merges the aggregate from the ``delta view" to the old view allowing us to push down on both branches of the tree.
On the left side, we reach the the stale view so we stop.
On the right side, we reach the aggregate query (count) and since \textsf{videoId} is in group by clause, we can push down the sampling.
Then, we reach a many-to-one (foreign key) join allowing us to push down the sampling to the ``many" relation \tbl{LogIns}.

\begin{figure}[t] \vspace{-2em}
\centering
 \includegraphics[scale=0.20]{figs/example_expression_tree_2.pdf} \vspace{-.25em}
 \caption{Applying the rules described in Section \ref{push}, we illustrate how to optimize the sampling of our example maintenance plan.  \label{exexpr2}}\vspace{-1.75em}
\end{figure}

In terms of increased efficiency, since both the aggregate and foreign key join are ``above" the sampling operator, they require less computation and less memory.



%\input{delta.tex}
\section{Sampling for Approximate Corrections}
In the previous section, we described how to do the correction using the full delta view.
However, this is just as expensive as incremental view maintenance.
In this section, we describe how we can couple sampling with the correction calculation to
save on I/O and communication costs. 

\subsection{Sampling for Select-Project and Foreign-Key Join Views}
We can extend our deterministic corrections to approximate corrections from a 
simple random sample $S_{\Delta V}$ of the delta view $S_{\Delta V}\subseteq\Delta\textbf{V}$. 
Recall that a simple random sample is uniform sample where every row $r\in\Delta V$
is in $S_{\Delta V}$ with equal probability $p$. For Select-Project
and Foreign-Key Join Views, this means we have to take a sample of
the updates and then apply the view definition to the sample of the
updates. Formally, for every record u inserted into the table, with
probability $p$, we include it in the sample $S$. Then, we take
the sample updates $S$ and apply the view definition forming $S_{\Delta V}$.
Therefore,
\[
c\cdot f(S_{\Delta V})\approx\epsilon
\]
The scaling constant $c$ for the SUM and COUNT queries
depends on the sample size and is $c = \frac{K}{N}$.
This estimate is guaranteed to be unbiased, that is, in expectation the answer is $\epsilon$.

\subsection{Cost Analysis for Select-Project and Foreign-Key Join Views}
Let $n$ be the number of inserted records, $v$ be the cardinality of the old view, $v'$ be the cardinality of the new view, $\delta_v$ be the cardinality of the delta view, and $p$ be the sampling ratio.  

\textbf{Scan of Updates: }
Both incremental maintenance and our proposed solution require at least one scan of the inserted records, and in both solutions we can
load the updates into memory once and amortize that I/O cost over all views. 

\textbf{Delta View: } For Select-project and foreign-key join views, incremental maintenance has processing cost of $n$ records where the predicate or the join has to be evaluated for each inserted record. Our approach has a cost of $np$ as we have to evaluate this only on our sample. 

\textbf{Refresh: } For Select-project and foreign-key join views, incremental maintenance has to insert $\delta_v$ rows while we have to insert only $p\delta_v$ records. If there is an outlier index, this cost increases to $p\delta_v + l$. 

\textbf{Query: } As incremental maintenance completely refreshes the view, the cost of processing a query on the view is at most $v'$. For Select-project and foreign-key join views, the processing cost is $v + p\delta_v$ and this is guaranteed to be less than $v'$. 

\subsection{Accuracy Analysis For Select-Project and Foreign-Key Join Views}
By the Central Limit Theorem, means of independent random variables converge 
to a Gaussian distribution.
We can apply this theorem to bound the approximation error in the correction $\epsilon$.
For $\epsilon = c\cdot f(S_{\Delta V})$, and a sampling ratio of $p$.
Recall, from the previous section that $f$ can be expressed as a sum of the function $\phi$ applied to each row.
Let $\phi(S_{\Delta V})$ be set of $\phi$ applied to each of rows in the sample delta view.
Then variance of the estimate is:
\[
c^2\frac{var(\phi(S_{\Delta V}))}{p|\Delta V|}
\] 
Since there is asymptotic convergence to a normal distribution, we can use this variance to bound the distance the true $\epsilon$:   
\[
\epsilon \pm \gamma \cdot c\frac{std(\phi(S_{\Delta V}))}{p|\Delta V|}
\]
We can set $\gamma$ to our desired normal confidence level (eg. 1.96 for 95\% confidence).
In other words, the accuracy of the estimate depends on the variance of the inserted records.

\subsection{Sampling for Aggregation Views}
In the previous section, we discussed how the delta view did not contain enough
information to calculate a correction.
Similarly, sampling to estimate the correction for queries on Aggregation Views
is more challenging.
We notice that in the refreshed view each GROUP BY key is unique, and
thus, to sample the refreshed view we have to sample by GROUP BY keys
in the inserted records. For each inserted record we apply a hash
to the cols in the GROUP BY clause, and then we take the result of
the hash modulo a sampling ratio to sample the table. The result is
that we ensure that every record with the same group by key is either
fully in the sample or not, thus none of the rows in the delta view
are approximate. 
Then, we refresh this sample delta view instead of the full view.
We can then join this delta view with the old view to approximate $\epsilon$.

Unlike before this is not a sample of a delta view, it is a sample of the entire view joined with the stale resuts.
Thus, we denote this sample as $S_{V}$.
Also, recall that unlike the other views, Aggregation Views did not have 
an additional proportionality constant for a full correction.
\[
f(S_{V})\approx\epsilon
\]
However, when we introduce sampling a scaling constant is now neccessary.
The scaling constant $c$ for the SUM and COUNT queries depends on the sample size and is $c = \frac{K}{N}$, 
but $c = 1$ for the AVG query.
For SUM, COUNT, AVG, and VAR, this estimate of $\epsilon$ is unbiased as before.

\subsection{Cost Analysis for Aggregation Views}
Let $n$ be the number of inserted records, $v$ be the cardinality of the old view, $v'$ be the cardinality of the new view, $\delta_v$ be the cardinality of the delta view, and $p$ be the sampling ratio.  

\textbf{Scan of Updates: }
As in the Select-Project and Foreign-Key Join views, both incremental maintenance and our proposed solution require at least one scan of the inserted records, and in both solutions we can load the updates into memory once and amortize that I/O cost over all views. 

\textbf{Delta View: }  For aggregation views, incremental maintenance has the processing cost of $n$ and in addition an aggregation cost of $\delta_v$ where aggregates for each of the groups have to be maintained. In contrast, for aggregation views, our approach has a cost of $p\delta_v$ as we sample the group by keys and an expected processing cost of $np$. 
For aggregation views, in a distributed environment, there are potentially additional communication costs as the updates may not be partitioned by the group by key.

\textbf{Refresh: } For aggregation views, the cost is a little bit more complicated as we have a combination of insertions into the view and updates to the view. 
Incremental maintenance has to refresh $\delta_v$ rows while we have to refresh $p\delta_v$ rows. 
If there is a join index, in constant time, we can determine which rows are new insertions and which correspond to rows already in the view.

The costs become higher in a distributed environment as we need to consider communication and query processing engines that rely on partitioned joins rather than indices.
For aggregation views, we want to partition the data by the group by key.
This allows a paritioned join which only requires communication (a shuffle operation) of the delta table.
Therefore, in incremental maintenance we have to communicate $\delta_v$ rows while our solution requires $p\delta_v$ rows.

\textbf{Query: } For aggregation views, the cost is $v + pv'$ where we calculate a correction by processing $pv'$ rows and correct an existing aggregation of $v$ records.

\subsection{Accuracy Analysis For Aggregation Views}
As in the Select-Project and Foreign-Key views the Central Limit Theorem can
be used to bound the approximation.
However, there are two cases with aggregation views: (1) updates to existing rows and (2) insertions into the view.
Let $S_{V}^{(i)}$ be the set differences for those rows in the sample materialized view that are updated; that is, represents how much each attribute 
changes, and let $S_{V}^{(i)}$ be the sample materialized view that correspond to inserted records.
Let $w_u$ by the fraction of the delta view which are updates to existing rows and $w_i$ be the fraction insertions into the view.

Recall, from the previous section that an aggregation function on the view $f$ can be expressed as mean of the function $\phi$ applied to each row.
Let $\phi(S_{V}^{(i)})$ be set of $\phi$ applied to each of rows that inserted, $\phi(S_{V}^{(i)})$ be the set of $\phi$ applied to the update differences, and $\phi(S_{V})$ be the union of the two sets.
As before, we can use the Central Limit Theorem to calculate the variance of the estimate and bound the approximation error.
The variance of a mixture of two distributions is:
\[
\sigma^2 = w_1((\mu_1-\mu)^2 + \sigma^2_1) + w_2((\mu_2-\mu)^2 + \sigma^2_2)
\]
Applying this in our setting, let us define the following values:
\[
\mu^{(i)}_{diff} = mean(\phi(S_{V}^{(i)})) - mean(\phi(S_{V}))
\]
\[
\mu^{(u)}_{diff} = mean(\phi(S_{V}^{(i)})) - mean(\phi(S_{V}))
\]
\[
\sigma^2_{(i)} = var(\phi(S_{V}^{(i)}))
\]
\[
\sigma^2_{(u)} = var(\phi(S_{V}^{(u)}))
\]
\[
\sigma^2_{diff} = w_i((\mu^{(i)}_{diff})^2 + \sigma_{(i)}) + w_u((\mu^{(u)}_{diff})^2 + \sigma_{(u)})
\] 
Therefore, the estimate variance is:
\[
c^2\frac{\sigma^2_{diff}}{|S_{V}|}
\]
And as before there is asymptotic convergence to a normal distribution, we can use this variance to bound the distance the true $\epsilon$:   
\[
\epsilon \pm \gamma \cdot c\frac{\sigma^2_{diff}}{|S_{V}|}
\]
We can set $\gamma$ to our desired normal confidence level (eg. 1.96 for 95\% confidence).

To interpret $\sigma^2_{diff}$, it not only has dependence on the variance of the inserted records, but also on the variance of the updates and the relative difference between the updates and inserted records. 
A worst case, would be if there is a 50-50 split between updates and insertions and changes to existing rows are in distribution very different than the new inserted rows in the materialized view. 


%\input{correction2.tex}
\section{Outlier Indexing}\label{outlier}
Sampling is known to be sensitive to outliers \cite{clauset2009power, chaudhuri2001overcoming}.
SVC may be sensitive to power-laws and other long-tailed distributions which are common in large datasets\cite{clauset2009power}.
Since outliers may occur very rarely, they are unlikely to be represented in a small sample. 
We address this problem using a technique called outlier indexing which has been applied in SAQP \cite{chaudhuri2001overcoming}.
The basic idea is that we create an index of outlier records and ensure that these records are included in the sample.
However, as this has not been explored in the materialized view setting there are new challenges in using this index for improved result accuracy.

\subsection{Indices on the Base Relations}
In SVC, we cannot index the sample in the same way as \cite{chaudhuri2001overcoming}. 
This is because we only materialize the sample view and we have no idea about outliers outside of this sample.
Instead, we propose building indices on the base relations.

The first step is that the user selects an attribute of any base relation to index and specifies a threshold $t$ and a size limit $k$.
In a single pass of updates (without maintaining the view), the index is built storing references to the records with attributes greater than $t$.
If the size limit is reached, the incoming record is compared to the smallest indexed record and if it is greater then we evict the smallest record.
The same approach can be extended to attributes that have tails in both directions by making the threshold $t$ a range, which takes the highest and the lowest values.
However, in this section, we present the technique as a threshold for clarity.

To select the threshold, there are many heuristics that we can use.
For example, we can use our knowledge about the dataset to set a threshold.
Or we can use prior information from the base table, a calculation which can be done in the background during the periodic maintenance cycles.
If our size limit is $k$, we can find the records with the top $k$ attributes in the base table as to set a threshold to maximally fill up our index. 
Then, the attribute value of the lowest record becomes the threshold $t$.

\subsection{Adding Outliers to the Sample}
Given this index, the next question is how we can use this information in our materialized views.
We ensure that any row in a materialized view that derives from an indexed record is guaranteed to be in the sample.
This problem is sort of an inverse to the efficient sampling problem.
We need to propagate the indices upwards through the query tree.

The next challenge is the outlier index must not require any additional effort to materialize.
We add the condition that the only eligible indices are ones on base relations that are being sampled (ie. we can push the hash operator down to that relation).
Therefore, in the same iteration as sampling, we can also test the threshold and add records to the outlier index.
We formalize the propagation property recursively. 
Every relation can have an outlier index which is a set of attributes and a set of records that exceed the threshold value on those attributes.

\begin{definition}
(OUTLIER INDEX PUSHUP). Define an outlier index to be a tuple of a set of attributes, relation, and a set of records $(A,R,O)$. The outlier index propagates upwards with the following rules:
\begin{itemize}\vspace{-.45em}
\item Base Relations: Outlier indices on base relations are pushed up only if that relation is being sampled.\vspace{-.45em}
\item $\sigma_{\phi}(R)$: Always propagates upwards \vspace{-.45em}
\item $\Pi_{p,(a_2,...,a_k)}(R)$: Propagates upwards if $A$ is in the projection, otherwise null.\vspace{-.45em}
\item $\bowtie_{\phi (r1,r2)}(R_1,R_2)$: New outlier index where $A=A_{r1} \cup A_{r2}$ and $O = \bowtie_{\phi (r1,r2)}(O_1,O_2)$.
\item $\gamma_{e}(R)$: New outlier index where $A=\{e\}$ and $O = \gamma_{e}(O)$.\vspace{-.45em}
\item $R_1 \cup R_2$: Propagates if $A_1 = A_2$, in which case $O = O_1 \cup O_2$.
\item $R_1 \cap R_2$: Propagates if $A_1 = A_2$, in which case $O = O_1 \cap O_2$.
\item $R_1 - R_2$: Propagates if $A_1 = A_2$, in which case $O = O_1 - O_2$.
\end{itemize}
\end{definition}
For all outlier indices that can propagate to the view (ie. the top of the tree), we get a final set $O$ of records. 
This gives us a deterministic set of records in addition to our sample which we can use in query processing.

\subsection{Query Processing with the Outlier Index} 
The outlier index has two uses: (1) we can query all the rows that correspond to outlier rows, 
and (2) we can improve the accuracy of our \emph{aggregation} queries.
To query the outlier rows, we can select all of the rows in the materialized view that are flagged as outliers, and these rows are guaranteed to be up-to-date.

We can also incorporate the outliers into our correction estimates.  
By guaranteeing that certain rows are in the index, we
have to merge two results: one over the outliers and one over the regular records.
For a given aggregation query, let $N$ be the count of records that satisfy the query's condition and $l$ be the number of outliers that satisfy the condition.
Let $v_{reg}$ be the query result for the regular records, and $v_{out}$ is the query result for outliers, then:

\[
 v = \frac{N-l}{N}v_{reg} + \frac{l}{N}v_{out}
\]

We can use this method to improve the accuracy of our correction estimates by calculating $\dans$ 
on the outliers and the regular records separately then averaging them together. 
%See \cite{chaudhuri2001overcoming} for additional query processing details.
\section{Extensions and Discussion}
In this section, we discuss some additional properties of our framework which
we excluded from the previous sections for clarity of presentation.

\subsection{Class of Aggregation Views}
There is a taxonomy of aggregation queries: distributive, holistic, and algebraic; refer to \cite{gray1997data} for details.
Likewise, the same taxonomy can be extended to materialized views.
Distributive queries require only a single parameter during the incremental refresh step, for example SUM and COUNT queries only need to
add the SUM from the delta view and the stale view.
Algebraic queries require a constant number of parameters, for example the AVG query requires the group count of the updates and the stale view before updating the aggregate.
We presented an approach geared towards aggregation views defined by distributive and algebraic queries (SUM,COUNT, AVG, MAX, MIN).
However, in the case of holisitc aggregates (eg. Median), we can still acheive performance gains.
For these functions, the refresh step of incremental maintenance may require the entire distribution and not just merging two aggregates.
For example, if we have a stale view defined by the Median, we have to know the entire distribution to keep it incrementally maintained.
One way to do this is to maintain a histogram for each group on the relevant attribute on the stale view, and do the same on the delta view.
Then, during the refresh we can merge the two histograms to update the median.
Even in this setting, Sampling can help reduce processing (less histograms to compare) and communication (less histograms to communicate). 

\subsection{Selection Queries}
While our approach is optimal for SUM, COUNT, and AVG queries, and is exact for Selection queries on the outlier index, there is a question about processing general selection queries.
For a Selection query, there are two possibilities: (1) the row is in the sample, and (2) the row in not in the sample.
For rows in the sample, we can get an exact result.
For rows not in the sample, we can bound the selection query using the Chebyshev inequality [?].
\begin{equation}
Pr(|X-\mu|\ge z\sigma)\le \frac{1}{z^2}
\end{equation}
Using the sample, we can estimate the average value and the standard deviation over all rows.
This, gives us a bound on the distribution of unsampled rows.
The value will not deviate from the mean by more than $\approx 4.5$ standard deviations with 95\% probability.
This bound is very loose and is not practical for many applications, however, this result is promising as it does indicate 
we can still acheive guarantees on general selection queries.
We defer caclulating a correction for selection queries to future work.

\section{Results}

\subsection{Experimental Setting}

\subsection{Query Correction}

\begin{figure*}[h]
\label{conviva}
\centering
 \includegraphics[width=\textwidth]{exp/exp1-conviva1.png}
 \caption{TODO}
\end{figure*}
\section{Related Work}
Addressing the cost of materialized view maintenance is the subject of many recent works with
techniques including reducing coordination \cite{bailis2014scalable}, using additional information about the view (eg. if the view is derived from linear algebraic operations) \cite{nikolic2014linview}, and notification-based systems \cite{percolator}.
The increased research focus parallels an major concern in industrial systems for incrementally updating pre-computed results and indices such as Google Percolator \cite{percolator}, LinkedIn \cite{qiao2013brewing}, and Twitter's Rainbird.
The steaming community has also studied the view maintenance problem \cite{abadi2003aurora,golab2011consistency, golab2012scalable, he2010comet, ghanem2010supporting}, in Spark Streaming they studied how they could exploit in-memory materialization \cite{zaharia2012discretized}, and in MonetDB they studied how ideas from columnar storage can be applied to enable real-time analytics \cite{liarou2012monetdb}.

Sampling has also been well studied in the context of query processing \cite{agarwal2013blinkdb, olken1993random, garofalakis2001approximate}.
However, we argue, that our application of sampling in this work has a fundementally different goal.
Prior work emphasizes sampling as a technique to reduce query execution time.
We, on the other hand, use sampling to reduce maintenance costs.
This is similar to the goals of load shedding studied in streaming databases \cite{tatbul2003load, rabkin2014aggregation}.
Babcok et al. studied load shedding in the context of predefined aggregate queries, however, did not support ad hoc queries on the streaming data \cite{babcock2004load}.

Sampling has also been studied in the context of materialized views.
Joshi and Jermaine \cite{joshi2008materialized} proposed using sampled materialized views.
This technique mirrors what we called SAQP in our evaluation.
Their focus, however, was not addressing incremental maintenance costs but rather operations on materialized views.
They proposed a tree-like data structure that could support queries on multiple materialized views and operations on these views.
Gibbons et al. studied the maintenace of approximate histograms \cite{gibbons1997fast, gibbons1998new}, which closely resemble aggregation materialized views.
They, however, did not consider queries on these histograms but took a holistic approach to analyze the error on the entire histogram.
We contrast our approach from those proposed in Joshi and Jermaine and Gibbons et al. since we do not estimate our query results directly from a sample.
We use the sample to learn how the updates affect the query results and then compensate for those changes.
Our experiments suggest that prior approaches (SAQP) are inefficient when updates are sparse and the maintenance batch is small compared to the base data.

Building off Gibbon et al. there are a variety of other works proposing storage efficent processing of aggregate queries on streams \cite{dobra2002processing, greenwald2001space} which are similar to materialized views. Furthermore, there is a close relationship between sampling and probabilistic databases, and view maintenance and selection in the context of probabilistic databases has also been studied \cite{re2007materialized}.

\section{Conclusion}
In this paper, we proposed a new approach to the staleness problem in materialized views.
We demonstrated how recent results from data cleaning, namely sampling, query correction, and outlier detection, can
allow for accurate query processing on stale views for a fraction of the cost of incremental maintenace.
We evaluated this approach on a single node and in a distributed environment and found that sampling can provide a flexible tradeoff 
between accuracy and performance.
In one of our end-to-end experiments with a 1TB log dataset from Conviva, we found that...\reminder{Find way to summarize these experiments}

Our results are promising and suggest many avenues for future work.
``Serving" machine learning models has been a recent hot topic of database research [?].
We believe there is a strong link between machine learning models and materialized views, and the principles of our approach could be applied
to streaming machine learning applications.
For example, we can sample a data stream and update a model (eg. via Stochastic Gradient Descent).

Another promising direction is to extend this approach to support a more general class of queries.
To support general selection queries, we will have to extend our query correction to predict and attribute value via a regression.
This is an ideal application for non-parametric regression techniques such as Gaussian Process Regression which do
not make strong assumptions about the distribution of the data.




\bibliographystyle{abbrv}
%\scriptsize
\fontsize{6.08pt}{6.4pt} \selectfont
\bibliographystyle{abbrv}
\bibliography{ref} 

\section{Appendix}

\subsection{Extensions}
\subsubsection{MIN and MAX}
\minfunc and \maxfunc fall into their own category since this is a canonical case where bootstrap fails.
We devise an estimation procedure that corrects these queries.
However, we can only achieve bound that has a slightly different interpretation than the confidence intervals seen before.
We can calculate the probability that a larger (or smaller) element exists in the unsampled view.
%Refer to the extended technical report for the details on \minfunc and \maxfunc \cite{technicalReport}.

We devise the following correction estimate for \maxfunc: (1) For all rows in both $S$ and $S'$, calculate the row-by-row difference, (2) let $c$ be the max difference, and (3) add $c$ to the max of the stale view.

We can give weak bounds on the results using Cantelli's Inequality.
If $X$ is a random variable with mean $\mu_x$ and variance $var(X)$, then the probability that $X$ is larger than a constant $\epsilon$ 
\[
\mathbb{P}(X \ge \epsilon + \mu_x ) \le \frac{var(X)}{var(X) + \epsilon^2}
\]
Therefore, if we set $\epsilon$ to be the difference between max value estimate and the average value, we can calculate the probability that we will see a higher value. 

The same estimator can be modified for \minfunc, with a corresponding bound:
\[
\mathbb{P}(X \le \mu_x - a )) \le \frac{var(x)}{var(x) + a^2}
\]
This bound has a slightly different interpretation than the confidence intervals seen before.
This gives the probability that a larger (or smaller) element exists in the unsampled view.


\vspace{-.25em}
\subsubsection{Select Queries}
In \svc, we also explore how to extend this correction procedure to Select queries.
Suppose, we have a Select query with a predicate:
\begin{lstlisting} [mathescape]
SELECT $*$ FROM View WHERE Condition(A);
\end{lstlisting}

We first run the Select query on the stale view, and this returns a set of rows.
This result has three types of data error: rows that are missing, rows that are falsely included, and rows whose values are incorrect.

As in the \sumfunc, \countfunc, and \avgfunc query case, we can apply the query to the sample of the up-to-date view.
From this sample, using our lineage defined earlier, we can quickly identify which rows were added, updated, and deleted.
For the updated rows in the sample, we overwrite the out-of-date rows in the stale query result.
For the new rows, we take a union of the sampled selection and the updated stale selection.
For the missing rows, we remove them from the stale selection.
To quantify the approximation error, we can rewrite the Select query as \countfunc to get an estimate of number of rows that were updated, added, or deleted (thus three ``confidence'' intervals).

\subsection{Additional Experiments}

\subsubsection{Aggregate View}
\label{exp-datacube}

\begin{figure}[t]
\centering
 \includegraphics[scale=0.15]{exp/msdc_1.pdf}
 \includegraphics[scale=0.15]{exp/msdc_2.pdf}
  %\includegraphics[scale=0.17]{exp/msdc_3.pdf}
   \caption{(a) In the aggregate view case, sampling can save significant maintenance time. (b) As the update size grows SVC tends towards an ideal speedup of 10x.\label{exp2-acc-sample}}
\end{figure}


\begin{figure}[t]
\centering
 \includegraphics[scale=0.17]{exp/msdc_3.pdf}
   \caption{We measure the accuracy of each of the roll-up aggregate queries on this view. For a 10\% sample size and 10\% update size, we find that SVC+Corr is more accurate than SVC+AQP and No Maintenance.\label{exp2-acc-sample2}}
\end{figure}



\begin{figure}[t]
\centering
\includegraphics[scale=0.17]{exp/msdc_4.pdf}
   \caption{For 1GB of updates, we plot the max error as opposed to the median error in the previous experiments. Even though updates are 10\% of the dataset size, some queries are nearly 80\% incorrect. SVC helps significantly mitigate this error. \label{exp2-max}}
\end{figure}

\begin{figure}[t]
\centering
  \includegraphics[scale=0.17]{exp/msdc_5.pdf}
  %\includegraphics[scale=0.20]{exp/msdc_6.pdf}
 \caption{We run the same experiment but replace the \sumfunc query with a median query. We find that similarly SVC is more accurate.\label{exp2-median} }
\end{figure}

In our next experiment, we evaluate an aggregate view use case similar to a data cube.
We generate a 10GB base TPCD dataset with skew $z=1$, and derive the base cube as a materialized view from this dataset.
We add 1GB of updates and apply SVC to estimate the results of all of the ``roll-up" dimensions.

\textbf{Performance: }
We observed the same trade-off as the previous experiment where sampling significantly reduces the maintenance time (Figure \ref{exp2-acc-sample}(a)).
It takes 186 seconds to maintain the entire view, but a 10\% sample can be maintained in 26 seconds.
As before, we fix the sample size at 10\% and vary the update size.
We similarly observe that SVC becomes more efficient as the update size grows (Figure \ref{exp2-acc-sample}(b)), and at an update size of 20\%  the speedup is 8.7x.

\textbf{Accuracy: }
In Figure \ref{exp2-acc-sample2}, we measure the accuracy of each of the ``roll-up" aggregate queries on this view.
That is, we take each dimension and aggregate over the dimension.
We fix the sample size at 10\% and the update size at 10\%.
On average SVC+Corr is 12.9x more accurate than the stale baseline and 3.6x more accurate than SVC+AQP (Figure \ref{exp2-acc-sample}(c)). 

Since the data cubing operation is primarily constructed by group-by aggregates, we can also measure the max error for each of the aggregates.
We see that while the median staleness is close to 10\%, for some queries some of the group aggregates have nearly 80\% error (Figure \ref{exp2-max}).
SVC greatly mitigates this error to less than 12\% for all queries.

\textbf{Other Queries: }
Finally, we also use the data cube to illustrate how SVC can support a broader range of queries outside of \sumfunc, \countfunc, and \avgfunc.
We change all of the roll-up queries to use the \textbf{median} function (Figure \ref{exp2-median}).
First, both SVC+Corr and SVC+AQP are more accurate as estimating the median than they were for estimating sums. 
This is because the median is less sensitive to variance in the data.

\subsubsection{Mini-batch Experiments}
\begin{figure}[t]
\centering
 \includegraphics[scale=0.14]{exp/con_1.pdf}
 \includegraphics[scale=0.14]{exp/con_2.pdf}
 \caption{(a) Spark RDDs are most efficient when updated in batches. As batch sizes increase the system throughput increases. (b) When running multiple threads, the throughput reduces. However, larger batches are less affected by this reduction. \label{conv-2}}
\end{figure}

We devised an end-to-end experiment simulating a real integration with periodic maintenance.
However, unlike the MySQL case, Apache Spark does not support selective updates and insertions as the ``views" are immutable.
A further point is that the immutability of these views and Spark's fault-tolerance requires that the ``views" are maintained synchronously.
Thus, to avoid these significant overheads, we have to update these views in batches.
Spark does have a streaming variant \cite{zaharia2012discretized}, however, this does not support the complex SQL derived materialized views used in this paper, and still relies on mini-batch updates.

SVC and IVM will run in separate threads each with their own RDD materialized view.
In this application, both SVC and IVM maintain respective their RDDs with batch updates.
In this model, there are a lot of different parameters: batch size for periodic maintenance, batch size for SVC, sampling ratio for SVC, and the fact that concurrent threads may reduce overall throughput.
Our goal is to fix the throughput of the cluster, and then measure whether SVC+IVM or IVM alone leads to more accurate query answers.

\textbf{Batch sizes:} In Spark, larger batch sizes amortize overheads better.
In Figure \ref{conv-2}(a), we show a trade-off between batch size and throughput of Spark for V2 and V5.
Throughputs for small batches are nearly 10x smaller than the throughputs for the larger batches. 

\textbf{Concurrent SVC and IVM:} Next, we measure the reduction in throughput when running multiple threads.
We run SVC-10 in loop in one thread and IVM in another.
We measure the reduction in throughput for the cluster from the previous batch size experiment.
In Figure \ref{conv-2}(b), we plot the throughput against batch size when two maintenance threads are running.
While for small batch sizes the throughput of the cluster is reduced by nearly a factor of 2, for larger sizes the reduction is
smaller.
As we found in later experiments (Figure \ref{conv-5}), larger batch sizes are more amenable to parallel computation since there was more idle CPU time.

\textbf{Choosing a Batch Size:}
The results in Figure \ref{conv-2}(a) and Figure \ref{conv-2}(b) show that larger batch sizes are more efficient, however, larger batch sizes also lead to more staleness.
Combining the results in Figure \ref{conv-2}(a) and Figure \ref{conv-2}(b), for both SVC+IVM and IVM, we get cluster throughput as a function of batch size.
For a fixed throughput, we want to find the smallest batch size that achieves that throughput for both.
For V2, we fixed this at 700,000 records/sec and for V5 this was 500,000 records/sec.
For IVM alone the smallest batch size that met this throughput demand was 40GB for both V2 and V5.
And for SVC+IVM, the smallest batch size was 80GB for V2 and 100GB for V5. 
When running periodic maintenance alone view updates can be more frequent, and when run in conjunction with SVC it is less frequent. 

We run both of these approaches in a continuous loop, SVC+IVM and IVM, and measure their maximal error during a maintenance period.
There is further a trade-off with the sampling ratio, larger samples give more accurate estimates however between SVC batches they go stale.
We quantify the error in these approaches with the max error; that is the maximum error in a maintenance period (Figure \ref{conv-4}).
These competing objective lead to an optimal sampling ratio of 3\% for V2 and 6\% for V5.
At this sampling point, we find that applying SVC gives results 2.8x more accurate for V2 and 2x more accurate for V5.

\begin{figure}[t]
\centering
 \includegraphics[scale=0.14]{exp/con_5.pdf}
 \includegraphics[scale=0.14]{exp/con_6.pdf}
 \caption{For a fixed throughput, SVC+Periodic Maintenance gives more accurate results for V2 and V5. \label{conv-4}} 
\end{figure}

\begin{figure}[t]
\centering
\includegraphics[width=\columnwidth]{exp/con_7.pdf}
 \caption{SVC better utilizes idle times in the cluster by maintaining the sample.\label{conv-5}} 
\end{figure}
To give some intuition on why SVC gives more accurate results, in Figure \ref{conv-5}, we plot the average CPU utilization of the cluster for both periodic IVM and SVC+periodic IVM. 
We find that SVC takes advantage of the idle times in the system; which are common during shuffle operations in a synchronous parallelism model.

In a way, these experiments present a worst-case application for SVC, yet it still gives improvements in terms of query accuracy.
In many typical deployments throughput demands are variable forcing maintenance periods to be longer, e.g., nightly.
The same way that SVC takes advantage of micro idle times during communication steps, it can provide large gains during controlled idle times when no maintenance is going on concurrently.

\subsection{Extended Proofs}

\subsection{Is Hashing Equivalent To RNG?}
In this work, we argue that hashing can be used for ``sampling" a relational expression.
However, from a complexity theory perspective, hashing is not equivalent to random number generation (RNG).
The existence of true one-way hash functions is a conjecture that would imply $P \ne NP$.
This conjecture is often taken as an assumption in Cryptography.
Of course, the ideal one-way hash functions required by the theory do not exist in practice. However, we find that existing hashes (e.g., linear hashes and SHA1) are sufficiently close to ideal that they can still take advantage of this theory. 
On the other hand, a SHA1 hash is nearly an order of magnitude slower but is much more uniform.
This assumption is called the Simple Uniform Hashing Assumption (SUHA) \cite{cormenintroduction}, and is widely used to analyze the performance of hash tables and hash partitioning.
There is an interesting tradeoff between the latency in computing a hash compared to its uniformity. For example, a linear hash stored procedure in MySQL is nearly as fast pseudorandom number generation that would be used in a TABLESAMPLE operator, however this hash exhibits some non-uniformity. 

\subsubsection{Hashing and Correspondence}
A benefit of deterministic hashing is that when applied in conjunction to the primary keys of a view, we get the Correspondence Property (Definition \ref{correspondence}) for free.
\begin{proposition}[Hashing Correspondence]
Suppose we have $S$ which is the stale view and $S'$ which is the up-to-date view.
Both these views have the same schema and a primary key $a$.
Let $\eta_{a, m}$ be our hash function that applies the hashing to the primary key $a$.
\[
\hat{S} = \eta_{a, m}(S)
\]
\[
\hat{S'} = \eta_{a, m}(S')
\]
Then, two samples $\hat{S'}$ and $\hat{S}$ correspond.
\end{proposition}
\begin{proof}
There are four conditions for correspondence:
\begin{itemize}
\item (1) Uniformity: $\widehat{S'}$ and $\widehat{S}$ are uniform random samples of $S'$ and $S$ respectively with a sampling ratio of $m$
\item (2) Removal of Superfluous Rows: $D = \{\forall s \in \widehat{S} \nexists s' \in S': s(u) = s'(u)\}$, $D \cap \widehat{S'} = \emptyset$ 
\item (3) Sampling of Missing Rows: $I = \{\forall s' \in \widehat{S'} \nexists s \in S: s(u) = s'(u)\}$, $\mathbb{E}(\mid I \cap \widehat{S'} \mid) = m\mid I \mid $ 
\item (4) Key Preservation for Updated Rows: For all $s\in \widehat{S}$ and not in $D$ or $I$, $s' \in \widehat{S}': s'(u) = s(u)$.
\end{itemize}
Uniformity is satisfied under by definition under SUHA (Simple Uniform Hashing Assumption).
Condition 2 is satisfied since if $r$ is deleted, then $r \not \in S'$ which implies that $r \not\in \hat{S'}$.
Condition 3 is just the converse of 2 so it is satisfied.
Condition 4 is satisfied since if $r$ is in $\hat{S}$ then it was sampled, and then since the primary key is consistent between $S$ and $S'$ it will also be sampled in $\hat{S'}$.
\end{proof}

\subsection{Theorem 1 Proof}
\begin{theorem}
Given a derived relation $R$, primary key $a$, and the sample $\eta_{a, m}(R)$.
Let $S$ be the sample created by applying $\eta_{a, m}$ without push down and 
$S'$ be the sample created by applying the push down rules to $\eta_{a, m}(R)$.
$S$ and $S'$ are identical samples with sampling ratio $m$.
\end{theorem}
\begin{proof}
We can prove this by induction.
The base case is where the expression tree is only one node, trivially making this true.
Then, we can induct considering one level of operators in the tree.
$\sigma, \cup, \cap, -$ clearly commute with hashing the key $a$ allowing for push down.
$\Pi$ commutes only if $a$ is in the projection.
For $\bowtie$, a sampling operator on $Q$ can be pushed down if $a$ is in either $k_r$ or $k_s$, or if there is a constraint that links $k_r$ to $k_s$.
There are two cases in which this happens a foreign-key relationship or an equality join on the same key.
For group by aggregates, if $a$ is in the group clause (i.e., it is in the aggregate) then a hash of the operand filters all rows that have $a$ which is sufficient to materialize the derived row.
It is provably NP-Hard to pushdown through a nested group by aggregate such as:
\begin{lstlisting}
SELECT c, count(1)
FROM ( 
       SELECT videoId, sum(1) as c FROM Log 
       GROUP BY videoId
     )
GROUP BY c
\end{lstlisting}
by reduction to a SUBSET-SUM problem.
\end{proof}

\subsection{More about the Hash Operator}
We defined a concept of tuple-lineage with primary keys.
However, a curious property of the deterministic hashing technique is that we can actually hash any attribute while retain the important statistical properties.
This is because a uniformly random sample of any attribute (possibly not unique) still includes every individual row with the same probability.  
A consequence of this is that we can push down the hashing operator through arbitrary equality joins (not just many-to-one) by hashing the join key.

We defer further exploration of this property to future work as it introduces new tradeoffs.
For example, sampling on a non-unique key, while unbiased in expectation, has higher variance in the size of the sample.
Happening to hash a large group may lead to decreased performance. 

Suppose our keys are duplicated $\mu_k$ times on average with variance $\sigma_k^2$, then the variance of the
sample size is for sampling fraction $m$:
\[m(1-m)\mu_k^2+(1-m)\sigma_k^2\]
This equation is derived from the formula for the variance of a mixture distribution.
In this setting, our sampling would have to consider this variance against the benefits of pushing the hash operator further down the query tree. 

\subsection{Experimental Details}

\subsubsection{Join View TPCD Queries}
In our first experiment, we materialize the join of lineitem and orders.
We treat the TPCD queries as queries on the view, and we selected 12 out of the 22 to include in our experiments.
The other 10 queries did not make use of the join.

\subsubsection{Conviva Views}
In this workload, there were annotated summary statistics queries, and we filtered for the most common types.
While, we cannot give the details of the queries, we can present some of the high-level characteristics of 8 summary-statistics type views. 
\begin{itemize} 
\item \textbf{V1.} Counts of various error types grouped by resources, users, date
\item \textbf{V2.} Sum of bytes transferred grouped by resource, users, date
\item \textbf{V3.} Counts of visits grouped by an expression of resource tags, users, date.
\item \textbf{V4.} Nested query that groups users from similar regions/service providers together then aggregates statistics
\item \textbf{V5.} Nested query that groups users from similar regions/service providers together then aggregates error types
\item \textbf{V6.} Union query that is filtered on a subset of resources and aggregates visits and bytes transferred
\item \textbf{V7.} Aggregate network statistics group by resources, users, date with many aggregates.
\item \textbf{V8.} Aggregate visit statistics group by resources, users, date with many aggregates.
\end{itemize}

\subsubsection{Data Cube Specification}
We defined the base cube as a materialized view:
\begin{lstlisting}
select
  sum(l_extendedprice * (1 - l_discount)) as revenue,
  c_custkey, n_nationkey,
  r_regionkey, L_PARTKEY
from
  lineitem, orders,
  customer, nation,
  region
where
  l_orderkey = o_orderkey and
  O_CUSTKEY = c_custkey and
  c_nationkey = n_nationkey and
  N_REGIONKEY = r_regionkey

group by
  c_custkey, n_nationkey, 
  r_regionkey, L_PARTKEY
\end{lstlisting}

Each of queries was an aggregate over subsets of the dimensions of the cube, 
with a \sumfunc over the revenue column.
\begin{itemize}
\item Q1. all
\item Q2. c\_custkey
\item Q3. n\_nationkey
\item Q4. r\_regionkey
\item Q5. l\_partkey
\item Q6. c\_custkey,n\_nationkey
\item Q7. c\_custkey,r\_regionkey
\item Q8. c\_custkey,l\_partkey
\item Q9. n\_nationkey, r\_regionkey
\item Q10. n\_nationkey, l\_partkey
\item Q11. c\_custkey,n\_nationkey, r\_regionkey
\item Q12. c\_custkey,n\_nationkey,l\_partkey
\item Q13. n\_nationkey,r\_regionkey,l\_partkey
\end{itemize}

When we experimented with the median query, we changed the \sumfunc to a median of the revenues.

\subsubsection{Table Of TPCD Queries 2}
We denormalize the TPCD schema and treat each of the 22 queries as views on the denormalized schema.
In our experiments, we evaluate 10 of these with SVC. 
Here, we provide a table of the queries and reasons why a query was not suitable for our experiments.
The main reason a query was not used was because the cardinality of the result was small.
Since we sample from the view, if the result was small eg. < 10, it would not make sense to apply SVC.
Furthermore, in the TPCD specification the only tables that are affected by updates are lineitem and orders; and queries that
do not depend on these tables do not change; thus there is no need for maintenance.

Listed below are excluded queries and reasons for their exclusion.
\begin{itemize}
\item Query 1. Result cardinality too small
\item Query 2. The query was static
\item Query 6. Result cardinality too small
\item Query 7. Result cardinality too small
\item Query 8. Result cardinality too small
\item Query 11. The query was static 
\item Query 12. Result cardinality too small
\item Query 14. Result cardinality too small
\item Query 15. The query contains an inner query, which we treat as a view.
\item Query 16. The query was static 
\item Query 17. Result cardinality too small
\item Query 19. Result cardinality too small
\item Query 20. Result cardinality too small
\end{itemize}




\end{document}
\section{Background}
Materialized views are stored query results that are used 
to optimize query processing [?].
However, pre-computed query results face the challenge of \emph{staleness} when applied
where the underlying tables are updating.
One solution is to recompute the materialized views when the table has been updated.
This can be very inefficient in the presence of small updates that hardly change the derived views.
Consequently, incremental maintenance of materialized views is well studied see [?]
for a survey of the approaches. 
Most incremental maintenance algorithms consists of two steps: calculating a ``delta" view,
and ``refreshing" the materialized view with the delta.
More formally, given a base relation $T$, a set of updates $U$,
and a view $\textbf{V}_{T}$:

\textbf{Calculate the Delta View- }
In this step, we apply the view definition to the updates and we call
the intermediate result a ``delta'' view:
\[
\Delta\textbf{V}
\]

This is also called a \emph{change propagation formula} in some literature,
especially on algebraic representations of incremental view maintenance.

\textbf{Refresh View- }
Given the ``delta'' view, we merge the results with the existing
view:
\[
\textbf{V}_{T}^{'}=refresh(\textbf{V}_{T},\Delta\textbf{V})
\] 
The details of the refresh operation depend on the view definition.
Refer to [?] for details.

%In this work, we address three types of materialized views: Select-Project, Aggregation, and
%Foreign-Key Join views; and four common aggregation queries on these
%views: SUM, COUNT, AVG, and VAR. We further analyze 
%only insertions into the database and defer analysis of updates
%and deletions for future work.

\subsection{Scheduling Maintenance}
While often less expensive than recomputing a materialized view,
incremental maintenance can still be computationally expensive.
Materialized views are growing larger and are more frequently 
implemented in distributed systems.
Furthermore, database systems face an increased velocity of incoming data which
can make infeasible to keep the views up-to-date.
Due to this cost, which we will refer to as the ``maintenance" cost, 
scheduling the refresh operation has been an important topic of research.

There are two principle types of scheduling strategies: immediate and deferred. 
In immediate maintenance, as soon as a record is updated, 
the change is propagated to any derived materialized view.
Immediate maintenance has an advantage that materialized view is always up-to-date, 
however it can be very expensive.
This scheduling strategy places a bottleneck when records are written reducing 
the available write-throughput for the database.
Furthermore, especially in a distributed setting, record-by-record 
maintenance cannot take advantage of the benefits of consolidating overheads by batching.
To address these challenges, deferred maintenance is alternative solution.
In deferred maintenance, the user often accepts some degree of staleness in 
the materialized view for additional flexibility for scheduling refresh operations.
For example, a user can update the materialized view 
nightly during times of low-activity in the system.

More sophisticated deferred scheduling schemes are also possible, refer to [?] for a full survey.
In particular, we highlight a technique called lazy maintenance which defers maintenance until such time
a query on the view requires a row to be up-to-date.
This technique is not exclusive and lazy maintenance can work in conjunction with immediate maintenance on a subset of rows [?].
While it ensures that query results are never stale, lazy maintenance potentially shifts much of the computational cost
to query execution time.

Immediate maintenance introduces a bottlneck on updates and lazy maintenance introduces a bottleneck during query exection,
and rapid collection of data can make these approches impractical.
Periodic maintainence allows for flexibility in scheduling to meet the resource constraints of the system and workload.
The problem is that between maintenance periods queries on the view are stale.
While, the user can set the maintenance period to control the expected staleness of the data.
In fact, in general, the relative error between the stale result and the up-to-date result is unbounded.

\subsection{Data Cleaning}
In the past, data cleaning research largely focused on improving query accuracy on dirty datasets.
This research focused on using rules or algorithms to remove or correct erroneous records [?].
However, recent work has considered the costs of data cleaning and how to budget effort.
The Data Wrangler project argues that rules to correct the entire dataset can be learned from a small set of training examples [?].
Similarly, the SampleClean project presents a query processing framework that cleans on a sample of data, and then bounds the results of aggregate queries on dirty datasets.

One key aspect of SampleClean is the tradeoff between data error and sampling error.
Cleaning a sample mitigates some of the error in the dataset, but sampling introduces uncertainty into the result.
There is a break-even point where the sample is sufficiently large that the mitigation of data error is more than the introduction
of sampling error.
This process allows the user to control the accuracy of the query with the sampling ratio as opposed to an unknown level of data error.
One of the algorithms in SampleClean, NormalizedSC, takes a dirty dataset and query and using a sample of clean data learns how
to correct a query so the result is ``clean".
This use of sampling is fundementally different from sampling as used in Sample-based Approximate Query Processing (SAQP).
In SAQP, the sample becomes the dataset and queries are applied directly to the sample.
However, in NormalizedSC, queries are still applied to the original unsampled and dirty base data.
The algorithm cleans a sample of data and then uses this information to correct the query result on the dirty data.

Since the ``correction" is derived from a sample, the correction is approximate but it turns out it bounded in error for SUM, COUNT, and AVG queries.
In comparision to SAQP, this approach gives more accurate results in datasets where errors are sparse since the correction will be small.
The idea of query correction is suited for the materialized view setting where if the views are very large updates may affect only a small fraction of rows in the view.
In this work, we apply a similar idea to use sampling to learn how the updates affect a given query, then use that information to correct the query so it is up-to-date.



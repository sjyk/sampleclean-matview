\section{Background}\label{sec-background}

\subsection{Incremental Maintenance Main Ideas}
Incremental maintenance of materialized views is well studied, see \cite{chirkova2011materialized} for a survey of the approaches. 
Most incremental maintenance algorithms consists of two steps: calculating a ``delta" view,
and ``refreshing" the materialized view with the delta.
More formally, given a base table $T$, a set of updates $U$,
and a view $\textbf{V}_{T}$:

\textbf{Calculate the Delta View- }
In this step, we apply the view definition to the updates $U$ and we call
the intermediate result a ``delta'' view:
\[
\Delta\textbf{V}
\]

This is also called a \emph{change propagation formula} in some literature,
especially on algebraic representations of incremental view maintenance.

\textbf{Refresh View- }
Given the ``delta'' view, we merge the results with the existing
view:
\[
\textbf{V}_{T}^{'}=refresh(\textbf{V}_{T},\Delta\textbf{V})
\] 
The details of the refresh operation depend on the view definition.
Refer to \cite{chirkova2011materialized} for details.

%In this work, we address three types of materialized views: Select-Project, Aggregation, and
%Foreign-Key Join views; and four common aggregation queries on these
%views: SUM, COUNT, AVG, and VAR. We further analyze 
%only insertions into the database and defer analysis of updates
%and deletions for future work.

\subsection{Maintenance Strategies}
There are two principle types of maintenance strategies: immediate and deferred. 
In immediate maintenance, as soon as the base table is updated, 
any derived materialized view is also updated.
Immediate maintenance has an advantage that queries on the materialized view are always up-to-date, 
however it can be very expensive.
This scheduling strategy places a bottleneck when records are written reducing 
the available write-throughput for the database.
Furthermore, especially in a distributed setting, record-by-record 
maintenance cannot take advantage of the benefits of consolidating communication overheads by batching.

To address these challenges, deferred maintenance is alternative solution.
The main idea of deferral is to avoid maintaining the view immediately and schedule an update at a more convenient time either in a pre-set way or adaptively.
In deferred maintenance approaches, the user often accepts some degree of staleness for additional flexibility for in scheduling.
For example, a user can update the materialized view nightly during times of low-activity in the system.
During these times, the system can take use more resources to process the updates without worrying about the affect on the throughput.
However, this also means that during the day the materialized view becomes increasingly stale as it was computed the night before.

More sophisticated deferred scheduling schemes are also possible.
In particular, we highlight a technique called lazy maintenance which defers maintenance until such time
a query on the view requires a row to be up-to-date \cite{zhou2007lazy}.
While it ensures that query results are never stale, lazy maintenance potentially shifts much of the computational cost to query execution time.

Immediate maintenance introduces a bottleneck on updates to the base table and lazy maintenance introduces a bottleneck during query execution,
and rapid updates can make these approaches impractical.
As a consequence, periodic maintenance or recalculation is often the most feasible solution.

\subsection{Data Cleaning}
Much of data cleaning research focuses on improving query accuracy on dirty datasets.
For example, designing rules or algorithms to remove or correct erroneous records \cite{rahm2000data}.
However, recent work has considered the costs of data cleaning and how to budget effort.
%The Data Wrangler project argues that rules to correct the entire dataset can be learned from a small set of training examples \cite{kandel2011wrangler}.
The SampleClean project presents a query processing framework that cleans a sample of data, and then bounds the results of aggregate queries on dirty datasets \cite{wang1999sample} with respect to the clean data.

This process allows the user to control the accuracy of the query with the sampling ratio as opposed to querying a dirty dataset with an unknown level of data error.
One of the algorithms in SampleClean, NormalizedSC, takes a dirty dataset and query and using a sample of clean data learns how
to correct a query so the result is ``clean".
Since the ``correction" is derived from a sample, the correction is approximate but it turns out it bounded in error for  \sumfunc, \countfunc, and \avgfunc queries.
Inspired by this solution, we apply a data cleaning cleaning approach to materialized views where we model staleness as a type of data error and budget our cleaning effort to get an approximate correction for aggregate queries (\sumfunc, \countfunc, and \avgfunc).

\subsection{SAQP}
Estimating the results of aggregate queries from samples has been
well studied in a field called Sample-based Approximate Query Processing
(SAQP). 
Our approach differs from SAQP as we look to
approximately correct a query rather than directly estimating the query result.
In other words, we use a sample of up-to-date data to understand how to correct for the
staleness. 
The SAQP approach to this problem, would be to
estimate the result directly from the maintained sample \cite{joshi2008materialized}.
We found that estimating
a correction and leveraging an existing deterministic result led
to lower variance results on real datasets (see Section \ref{exp}). 



\section{Extensions}
\subsection{Select Queries}\label{sec:sel}
We can correct stale Selection queries with SVC.
For \fjview and \spview, a stale selection query is missing rows.
To correct this result, we can apply the query to the update pattern sample, we get a sample of records that satisfy the predicate.
Then, we can take a union of the sampled selection and the stale selection.
To quantify the approximation error, we can rewrite the Select query as \countfunc to get an estimate of the up-to-date size of the result; giving us a bound on the number of missing rows.

For \aggview, stale selection queries can also have out-of-date rows as well as missing rows.
In this case, we can still apply the query to the update pattern sample and get a sample of rows to update and new rows to insert.
For the updated rows in the sample, we overwrite the out-of-date rows in the stale query result.
To quantify the approximation error, we can give two bounds.
As before, we can bound the number of missing rows in the result with a \countfunc query.
We further can bound the number of existing rows that are not up-to-date with a \countfunc query.

%Of course, this approach is not suited for highly selective queries which are unlikely to have sufficient representation in a sample.
In comparison to no maintenance, this approach gives a less stale result.
SAQP is limited in its support of Selection queries.
For example, if we had a 1\% sample, we could only get 1\% of the rows in the result.
In comparison, for a 1\% sample, we take advantage of the existing stale result allowing us to combine to old result with 1\% of the newly inserted records; thus including a much greater portion of the result rows.

\subsection{Deletions}\label{sec:del}
\reminder{TODO}

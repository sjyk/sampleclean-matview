\section{Analysis}

\subsection{Cost Analysis}
Let $n$ be the number of inserted records, $v$ be the cardinality of the old view, $v'$ be the cardinality of the new view, $\delta_v$ be the cardinality of the delta view, $p$ be the sampling ratio, and $l$ be the size of the outlier index.  

\textbf{Scan of Updates: }
Both incremental maintenance and our proposed solution require at least one scan of the inserted records, and in both solutions we can
load the updates into memory once and amortize that I/O cost over all views. 
If a user chooses to add an outlier index on an attribute in the updates building this index also requires a single scan of the inserted records.
However, as the index is based on the base relation and not the delta view this can be populated when the record is inserted.

\textbf{Delta View: } For Select-project and foreign-key join views, incremental maintenance has processing cost of $n$ records where the predicate or the join has to be evaluated for each inserted record. Our approach has a cost of $np$ as we have to evaluate this only on our sample. For aggregation views, incremental maintenance has the processing cost of $n$ and in addition an aggregation cost of $\delta_v$ where aggregates for each of the groups have to be maintained. In contrast, for aggregation views, our approach has a cost of $p\delta_v$ as we sample the group by keys and an expected processing cost of $np$. 
If there is an outlier index, the cost increases additively to $p\delta_v+l$.
For aggregation views, in a distributed environment, there are potentially additional communication costs as the updates may not be partitioned by the group by key.

\textbf{Refresh: } Let us first analyze the refresh step in a single node system with a join index. For Select-project and foreign-key join views, incremental maintenance has to insert $\delta_v$ rows while we have to insert only $p\delta_v$ records. If there is an outlier index, this cost increases to $p\delta_v + l$. For aggregation views, the cost is a little bit more complicated as we have a combination of insertions into the view and updates to the view. 
Incremental maintenance has to refresh $\delta_v$ rows while we have to refresh $p\delta_v$ rows. 
If there is a join index, in constant time, we can determine which rows are new insertions and which correspond to rows already in the view.

The costs become higher in a distributed environment as we need to consider communication and query processing engines that rely on partitioned joins rather than indices.
For aggregation views, we want to partition the data by the group by key.
This allows a paritioned join which only requires communication (a shuffle operation) of the delta table.
Therefore, in incremental maintenance we have to communicate $\delta_v$ rows while our solution requires $p\delta_v$ rows.

\textbf{Query: } As incremental maintenance completely refreshes the view, the cost of processing a query on the view is at most $v'$. In our approach, we use the old view, sample, and outlier index to process a query. For Select-project and foreign-key join views, the processing cost is $v + p\delta_v + l$ and this is guaranteed to be less than $v'$. For aggregation views, the cost is $v + pv'+l$ where we calculate a correction by processing $pv'+l$ rows and correct an existing aggregation of $v$ records.

\subsection{Variance Analysis}
We will first analyze the variance of the estimate $\epsilon$ without the outlier index. 
In Section ?, we discussed how the variance of $\epsilon$ is the variance of the difference set.
For Select-project and foreign-key join views, this variance can be analytically derived from the variance formula for statistical mixtures [?].
\begin{equation}
f(\Delta V)\frac{\delta_v}{v'}(1-\frac{\delta_v}{v'})+var(\Delta V)\frac{\delta_v}{v'}
\end{equation}
For Aggregation views, the accuracy depends on both the variance of the inserted rows and the updates to existing rows. 
<TODO>

\subsection{Optimality of Corrections}
<TODO>
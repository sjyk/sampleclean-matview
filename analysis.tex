\section{Extensions and Discussion}
In this section, we discuss some additional properties of our framework which
we excluded from the previous sections for clarity of presentation.

\subsection{Class of Aggregation Views}
There is a taxonomy of aggregation queries: distributive, holistic, and algebraic; refer to \cite{gray1997data} for details.
Likewise, the same taxonomy can be extended to materialized views.
Distributive queries require only a single parameter during the incremental refresh step, for example SUM and COUNT queries only need to
add the SUM from the delta view and the stale view.
Algebraic queries require a constant number of parameters, for example the AVG query requires the group count of the updates and the stale view before updating the aggregate.
We presented an approach geared towards aggregation views defined by distributive and algebraic queries (SUM,COUNT, AVG, MAX, MIN).
However, in the case of holisitc aggregates (eg. Median), we can still acheive performance gains.
For these functions, the refresh step of incremental maintenance may require the entire distribution and not just merging two aggregates.
For example, if we have a stale view defined by the Median, we have to know the entire distribution to keep it incrementally maintained.
One way to do this is to maintain a histogram for each group on the relevant attribute on the stale view, and do the same on the delta view.
Then, during the refresh we can merge the two histograms to update the median.
Even in this setting, Sampling can help reduce processing (less histograms to compare) and communication (less histograms to communicate). 

\subsection{Selection Queries}
While our approach is optimal for SUM, COUNT, and AVG queries, and is exact for Selection queries on the outlier index, there is a question about processing general selection queries.
For a Selection query, there are two possibilities: (1) the row is in the sample, and (2) the row in not in the sample.
For rows in the sample, we can get an exact result.
For rows not in the sample, we can bound the selection query using the Chebyshev inequality [?].
\begin{equation}
Pr(|X-\mu|\ge z\sigma)\le \frac{1}{z^2}
\end{equation}
Using the sample, we can estimate the average value and the standard deviation over all rows.
This, gives us a bound on the distribution of unsampled rows.
The value will not deviate from the mean by more than $\approx 4.5$ standard deviations with 95\% probability.
This bound is very loose and is not practical for many applications, however, this result is promising as it does indicate 
we can still acheive guarantees on general selection queries.
We defer caclulating a correction for selection queries to future work.

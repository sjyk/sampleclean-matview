\section{Outlier Indexing}

An application of particular interest for up-to-date query results
is outlier detection. When we have growing datasets, for example activity
logs, we may want to know which records correspond to abnormal activity.
Up-to-date query results, as opposed to long periods of staleness,
have to potential to detect these outliers quickly. However, as we
use sampling to processes aggregate queries on the views, this has
a potential to mask outliers. Our framework can be extended to guarantee
that outliers records will be incorporated into the sample. In this
section, we discuss how not only are these outliers themselves of
interest but that these outliers give information about the distribution
of an column, and can greatly improve the accuracy of estimates.


\subsection{A Model For Outlier Indexing}

We propose the following model for detecting and indexing outliers.
When creating a view, the user specifies an attribute in the base
relation to ``outlier index''. What this means is the that the records
with the $l$ largest attribute values are guaranteed to be included
in the sample view. In this outlier model, we detect outliers in updates
with a single pass and without having to build the entire delta table.
For aggregation views, for the records that are indexed as outliers,
we simply add those group by keys to the sample.


\subsection{Query Processing with the Outlier Index}

We can incorporate the outliers into our estimates of the correction
$\epsilon$. By guaranteeing that certain rows are in the index, we
have to merge a deterministic result (set of outlier rows) with the
estimate. One way to think of this is that we have $\epsilon$ is
calculated from the set of records that are not outliers. Let $|V_{T}^{'}|$
be the size of the updated view, $l$ be the number of rows in the
outlier index, and let $D_{o}$ be the set of differences (as defined
in Section ?) for the rows in the outlier index. We can update $\epsilon$
with the outlier information by:
\[
\frac{|V_{T}^{'}|}{|V_{T}^{'}|-l}\epsilon+c\cdot f(D_{o})
\]


\subsection{Increased Accuracy For Heavy-Tailed Distributions}

This outlier indexing procedure can greatly increase the accuracy
of estimates where the set of difference is heavy tailed. This approach
has been well studied in AQP {[}?{]} and is called truncation in Statistics
{[}?{]}. The intuition is that by removing the tail, you are reducing
the variance of the distribution, and thus, making it easier to estimate
an aggregate from a sample.
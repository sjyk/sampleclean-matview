\section{Outlier Indexing}\label{outlier}
The presented approach may be sensitive to long-tailed distributions and power-laws which are common in large datasets\cite{clauset2009power}.
Sampling may also hide any outliers, records with abnormally large or small attribute values.
Since oultiers may occur very rarely, they are unlikely to be represented in a small sample.
We address this problem using a technique called outlier indexing which has been applied in SAQP \cite{chaudhuri2001overcoming}.
The user specifies a threshold $t$ and an attribute on one of the base tables.
If a record has an attribute that exceeds $t$ it is added to the index.
We ensure that when we sample the update patterns for views that these records are represented in the sample.
The user can additionally place a size limit on the outlier index $k$.
The same approach can be extended to attributes that have tails in both directions by making the threshold $t$ a range, which takes the highest and the lowest values.
However, in this section, we present the technique as a threshold for clarity.

\subsection{Building The Outlier Index}
The first step is that the user selects an attribute of the base table to index and specifies a threshold $t$ and a size limit $k$.
In a single pass of updates, the index is built storing references to the records with attributes greater than $t$.
If the size limit is reached, the incoming record is compared to the smallest indexed record and if it is greater then we evict the smallest record.

To select the threshold, we can use prior information from the base table.
This approach is, of course, a heuristic which assumes that the updates are distributionally similar to the 
base table. 
This can be done in the background during the periodic maintenance cycles.
If our size limit is $k$, we can find the records with the top $k$ attributes in the base table as to set a threshold to maximally fill up our index. 
Then, the attribute value of the lowest record becomes the threshold $t$.

\subsection{Adding Outliers to the Sample}
Since the outlier indices index the base table, all of the materialized views in the system share a common set of these indices.
We discuss how to ensure that these records are added to the sample and what overhead this introduces.
For Select-Project and Foreign-Key Join views, we sample updates to view.
In the same pass as the sample, we can test each record against the outlier indices. 
If the record exists in any of the indices it is added to the sample with a flag indicating that it is an outlier.

For aggregation views, we sample the view by taking a hash of the group by keys.
If a record is in the outlier index, we must ensure that all records with its group by key are also added to the sample.
To achieve this, we have to select all of the distinct group by keys in the outlier index and add those aggregates to the sample.
As before, these records are marked with a flag denoting they are outliers.

Outlier indexing adds additional overhead since for Select-Project and Join views it adds the overhead of a hash table lookup for each record, and for Aggregation views it further requires a scan of the entire index.
However, we envision that in most datasets the outlier index will be very small making this overhead negligible.
In our experiments, we show that even a very small outlier index (less than .01\% of records) is sufficient to greatly improve the accuracy of
correction estimates.

\subsection{Query Processing with the Outlier Index} 
The outlier index has two uses: (1) we can answer \emph{selection} queries on the outliers, 
and (2) we can improve the accuracy of our \emph{aggregation} queries.
For selection queries, we must simply check to see if the record satisfying the query is in the outlier index.
While the outlier index is small, outliers (and their derived materialized view rows) are often the records that
we want to query.

We can also incorporate the outliers into our correction estimates. 
By guaranteeing that certain rows are in the index, we
have to merge a deterministic result (set of outlier rows) with the
estimate. One way to think of this is that we have $\dans$ is
calculated from the set of records that are not outliers. Let $\dans_{cnt} + \ans_{cnt}$
be the size of the updated view, and $l$ be the number of rows in the
outlier index. Furthermore, let $\dans(o)$ be the delta answer calculated on only the outliers.
We can update $\dans$ with the outlier information by:
\[
\frac{\dans_{cnt} + \ans_{cnt}-l}{\dans_{cnt} + \ans_{cnt}}\ans+\frac{l}{\dans_{cnt} + \ans_{cnt}}\dans(o)
\]
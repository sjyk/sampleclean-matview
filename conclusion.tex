\section{Related Work}\label{related}

Addressing the cost of materialized view maintenance is the subject of many recent works with
focus on various perspectives including complex analytical queries~\cite{nikolic2014linview}, transaction~\cite{bailis2014scalable}, and physical design~\cite{lefevre2014opportunistic}.
The increased research focus parallels a major concern in industrial systems for incrementally updating pre-computed results and indices such as Google Percolator~\cite{percolator} and Twitter's Rainbird~\cite{rainbird}.
The steaming community has also studied the view maintenance problem \cite{abadi2003aurora,golab2011consistency, golab2012scalable, he2010comet, ghanem2010supporting}. In Spark Streaming they studied how they could exploit in-memory materialization~\cite{zaharia2012discretized}, and in MonetDB they studied how ideas from columnar storage can be applied to enable real-time analytics \cite{liarou2012monetdb}.


Sampling has been well studied in the context of query processing~\cite{agarwal2013blinkdb, olken1993random, garofalakis2001approximate}. Particularly, a similar idea of sampling from updates has also been applied in stream processing~\cite{tatbul2003load, Garofalakis, rabkin2014aggregation}. But none of these works studied how to sample update patterns w.r.t materialized views and how to use the sample update patterns to correct stale query results.
%However, we argue, that our application of sampling in this work has a fundamentally different goal.
%Prior work emphasizes sampling as a technique to reduce query execution time.
%We, on the other hand, use sampling to reduce maintenance costs.
%This is similar to the goals of load shedding studied in streaming databases .
%Babcok et al. studied load shedding in the context of predefined aggregate queries, however, did not support ad hoc queries on the streaming data \cite{babcock2004load}.

Sampling has also been studied in the context of materialized views~\cite{joshi2008materialized,DBLP:conf/icde/OlkenR92}.
These techniques mirror what we called SAQP in our evaluation.
%Their focus, however, was not addressing incremental maintenance costs but rather operations on materialized views.
%They proposed a tree-like data structure that could support queries on multiple materialized views and operations on these views.
Gibbons et al. studied the maintenance of approximate histograms~\cite{gibbons1997fast}, which closely resemble aggregation materialized views.
They, however, did not consider queries on these histograms but took a holistic approach to analyze the error on the entire histogram.
We contrast our approach from those proposed since we do not estimate our query results directly from a sample.
We use the sample to learn how the updates affect the query results and then compensate for those changes.
Our experiments suggest that prior approaches (SAQP) are inefficient when updates are sparse and the maintenance batch is small compared to the base data.

There are a variety of other works proposing storage efficient processing of aggregate queries on streams \cite{dobra2002processing, greenwald2001space} which are similar to materialized views. Furthermore, there is a close relationship between sampling and probabilistic databases, and view maintenance and selection in the context of probabilistic databases has also been studied \cite{re2007materialized}.
Srinivasan and Carey studied compensation for concurrency errors that happen during query processing \cite{srinivasanC92}\reminder{Differentiate with our work}.
\section{Conclusion and Future Work}\label{conclusion}
In this paper, we propose a new approach to the staleness problem in materialized views.
We demonstrate how recent results from data cleaning, namely sampling, query correction, and outlier detection, can
allow for accurate query processing on stale views for a fraction of the cost of incremental maintenance. 
We evaluate this approach on a single node and in a distributed environment and find that sampling can provide a flexible tradeoff 
between accuracy and performance.

Our results are promising and suggest many avenues for future work.
We are interested in extending this approach to support a more general class of queries.
To support general selection queries, we will have to extend our query correction to predict and attribute value via a regression.
This is an ideal application for non-parametric regression techniques such as Gaussian Process Regression which do
not make strong assumptions about the distribution of the data.
This work also has applications in machine learning.
We believe there is a strong link between pre-computed machine learning models and materialized views, and the principles of our approach could be applied
to build fast, approximate streaming machine learning applications.




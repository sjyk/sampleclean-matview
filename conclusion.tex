\vspace{-.75em}
\section{Related Work}\label{related}
\vspace{-.25em}
Addressing the cost of materialized view maintenance is the subject of many recent papers, which
focus on various perspectives including complex analytical queries~\cite{nikolic2014linview}, transactions~\cite{bailis2014scalable}, and physical design~\cite{lefevre2014opportunistic}.
The increased research focus parallels a major concern in industrial systems for incrementally updating pre-computed results and indices such as Google Percolator~\cite{percolator} and Twitter's Rainbird~\cite{rainbird}.
The streaming community has also studied the view maintenance problem \cite{abadi2003aurora,golab2011consistency, golab2012scalable, he2010comet, ghanem2010supporting, KrishnamurthyFDFGLT10}. In Spark Streaming, Zaharia et~al. studied how they could exploit in-memory materialization~\cite{zaharia2012discretized}, and in MonetDB, Liarou et~al. studied how ideas from columnar storage can be applied to enable real-time analytics \cite{liarou2012monetdb}.
These works focus on correctness, consistency, and fault tolerance of materialized view maintenance.
SVC proposes an alternative model where we allow approximation error (with guarantees) for queries on materialized views for vastly reduced maintenance time.
In many decision problems, exact results are not needed as long as the probability of error is boundable. 

Sampling has been well studied in the context of query processing~\cite{AgarwalMPMMS13, olken1993random, garofalakis2001approximate}. 
Both the problems of efficiently sampling relations \cite{olken1993random} and processing complex queries \cite{agarwalknowing}, have been well studied. 
In SVC, we look at a new problem, where we efficiently sample from a maintenance strategy, a relational expression that updates a materialized view.
We generalize uniform sampling procedures to work in this new context using lineage \cite{DBLP:journals/vldb/CuiW03} and hashing.
We further extended the work in \cite{AgarwalMPMMS13, agarwalknowing} to consider how we can better leverage existing deterministic results by estimating a ``correction" rather than estimating query results. 
Srinivasan and Carey studied a problem related to query correction which they called compensation-based query processing \cite{srinivasanC92}.
This work was applied in the context of concurrency control.
However, this work did not consider applications when the correction was applied to a sample as in SVC.
The sampling in SVC introduces new challenges such as sensitivity to outliers, questions of bias, and estimate optimality. 

In the context of materialized view maintenance, sampling has primarly been primarily been studied from the perspective of maintaining samples \cite{DBLP:conf/icde/OlkenR92}.
Similarly, in~\cite{joshi2008materialized}, Joshi and Jermaine studied indexed materialized views that are amenable to random sampling.
While similar in spirit (queries on the view are approximate), this work does not consider the cost of maintaining the materialized view only the cost of fast, approximate queries on the view.
Nirkhiwale et al. \cite{DBLP:journals/pvldb/NirkhiwaleDJ13}, studied an algebra for sampling in aggregate queries.
They studied how to build query plans for aggregate queries (with nested subqueries and joins) that involved a Generalized Uniform Sampling (GUS) primitive.
This work did use lineage and similar to our work defined the GUS primitive as a random selection over the primary key.
This is similar to our model where we have a materialized view and aggregate queries on the materialized view.
However, they did not discuss the use of hashing to efficiently implement this primitive, and restricted their analysis to the \sumfunc query.
The problem setting was also very different where the objective is efficient planning of aggregate queries with sampling operators and not efficient updates of materialized views.

Sampling has been explored in the streaming community, and a similar idea of sampling from incoming updates has also been applied in stream processing~\cite{tatbul2003load, Garofalakis, rabkin2014aggregation}.
While some of these works studied problems similar to materialization, for example, the JetStream project (Rabkin et al.) looks at how sampling can help with real-time analysis of aggregates.
None of these works formally studied the class views that can benefit from sampling or formalized queries on these views.
However, there are ideas from Rabkin et al. that could be applied in SVC in future work, for example, their description of coarsening operations in aggregates is very similar to our experiments with the ``roll-up" queries in aggregate views.
There are a variety of other efforts proposing storage efficient processing of aggregate queries on streams \cite{dobra2002processing, greenwald2001space} which is similar to our problem setting and motivation.

Finally, the theory community has has studied related problems.
Gibbons et al. studied the maintenance of approximate histograms~\cite{gibbons1997fast}, which closely resemble aggregate materialized views.
This work did not model queries on the approximate histograms as in SVC.
Furthemore, the goals of this line work (including techniques such as sketching and approximate counting) has been to reduce the required storage, not to reduce the required update time.
There is also close relationship between sampling and probabilistic databases, and view maintenance and selection in the context of probabilistic databases have also been studied \cite{re2007materialized}.
While this line work studies query processing on probabilistic views, it does not study their maintenance as in SVC.

\vspace{-1em}
\section{Conclusion and Future Work}\label{conclusion}
\vspace{-.3em}
In this paper, we propose a new approach to the staleness problem in materialized views.
We demonstrate how recent results from data cleaning, namely sampling, query correction, and outlier detection, can
allow for accurate query processing on stale views for a fraction of the cost of incremental maintenance. 
We evaluate this approach on a single node and in a distributed environment and find that SVC can correct stale query results 
with orders of magnitude less cost than full incremental maintenance.

Our results are promising and suggest many avenues for future work.
In particular, we are interested in deeper exploration of the multiple view setting.
Here, given a storage constraint and throughput demands, we can optimize sampling ratios over all views.
We are also interested in the possibility of sharing computation between materialized views and maintenance on views derived from other views.
We also believe there is a strong link between pre-computed machine learning models and materialized views, and the principles of our approach could be applied to build fast, approximate streaming machine learning applications.




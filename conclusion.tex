\vspace{-1em}
\section{Related Work}\label{related}
\vspace{-.3em}

\iffalse
Addressing the cost of materialized view maintenance is the subject of many recent papers, which
focus on various perspectives including complex analytical queries~\cite{nikolic2014linview}, transactions~\cite{bailis2014scalable}, and physical design~\cite{lefevre2014opportunistic}.
The increased research focus parallels a major concern in industrial systems for incrementally updating pre-computed results and indices such as Google Percolator~\cite{percolator} and Twitter's Rainbird~\cite{rainbird}.
The streaming community has also studied the view maintenance problem \cite{abadi2003aurora,golab2011consistency, golab2012scalable, he2010comet, ghanem2010supporting, KrishnamurthyFDFGLT10}. In Spark Streaming, Zaharia et~al. studied how they could exploit in-memory materialization~\cite{zaharia2012discretized}, and in MonetDB, Liarou et~al. studied how ideas from columnar storage can be applied to enable real-time analytics \cite{liarou2012monetdb}.
These works focus on correctness, consistency, and fault tolerance of materialized view maintenance.
\svc proposes an alternative model where we allow approximation error (with guarantees) for queries on materialized views for vastly reduced maintenance time.
In many decision problems, exact results are not needed as long as the probability of error is boundable. 
\fi

Sampling has been well studied in the context of query processing~\cite{AgarwalMPMMS13, olken1993random, garofalakis2001approximate}. 
Both the problems of efficiently sampling relations \cite{olken1993random} and processing complex queries \cite{agarwalknowing}, have been well studied. 
In \svc, we look at a new problem, where we efficiently sample from a maintenance strategy, a relational expression that updates a materialized view.
We generalize uniform sampling procedures to work in this new context using lineage \cite{DBLP:journals/vldb/CuiW03} and hashing.
We look the problem of approximate query processing \cite{AgarwalMPMMS13, agarwalknowing} from a different perspective by estimating a ``correction'' rather than estimating query results. 
Srinivasan and Carey studied a problem related to query correction which they called compensation-based query processing \cite{srinivasanC92} for concurrency control.
%This work was applied in the context of concurrency control.
However, this work did not consider applications when the correction was applied to a sample as in \svc.
%The sampling in \svc introduces new challenges such as sensitivity to outliers, questions of bias, and estimate optimality. 

In the context of materialized view maintenance, sampling has primarily been studied from the perspective of maintaining samples \cite{DBLP:conf/icde/OlkenR92}.
Similarly, in~\cite{joshi2008materialized}, Joshi and Jermaine studied indexed materialized views that are amenable to random sampling.
While similar in spirit (queries on the view are approximate), the goal of this work was to optimize query processing not address the cost of incremental maintenance.
There has been work using sampled views in a limited context of cardinality estimation \cite{larson2007cardinality}, which is the special case of our framework, namely, the \countfunc query.
Nirkhiwale et al. \cite{DBLP:journals/pvldb/NirkhiwaleDJ13}, studied an algebra for sampling in aggregate queries.
They studied how to build query plans for aggregate queries (with nested subqueries and joins) that involved a Generalized Uniform Sampling (GUS) primitive.
This work did use lineage and similar to our work defined the GUS primitive as a random selection over the primary key.
This is similar to our model where we have a materialized view and aggregate queries on the materialized view.
However, they did not discuss the use of hashing to efficiently implement this primitive, and restricted their analysis to the \sumfunc query.
The problem setting was also very different where the objective is efficient planning of aggregate queries with sampling operators and not efficient updates of materialized views.

Sampling has been explored in the streaming community, and a similar idea of sampling from incoming updates has also been applied in stream processing~\cite{tatbul2003load, Garofalakis, rabkin2014aggregation}.
While some of these works studied problems similar to materialization, for example, the JetStream project (Rabkin et al.) looks at how sampling can help with real-time analysis of aggregates.
None of these works formally studied the class views that can benefit from sampling or formalized queries on these views.
%However, there are ideas from Rabkin et al. that could be applied in \svc in future work, for example, their description of coarsening operations in aggregates is very similar to our experiments with the ``roll-up'' queries in aggregate views.
There are a variety of other efforts proposing storage efficient processing of aggregate queries on streams \cite{dobra2002processing, greenwald2001space} which is similar to our problem setting and motivation.

Finally, the theory community has has studied related problems.
There has been work on the maintenance of approximate histograms, synopses, and sketches ~\cite{gibbons1997fast, DBLP:journals/ftdb/CormodeGHJ12}, which closely resemble aggregate materialized views.
This work did not model queries on the approximate data structures as in \svc.
Furthermore, the goals of this line work (including techniques such as sketching and approximate counting) has been to reduce the required storage, not to reduce the required update time.
%There is also close relationship between sampling and probabilistic databases, and view maintenance and selection in the context of probabilistic databases have also been studied \cite{re2007materialized}.
%While this line work studies query processing on probabilistic views, it does not study their maintenance as in \svc.
\vspace{-1em}
\section{Limitations and Opportunities}\vspace{-.3em}
\svc proposes a new approach for accurate query processing with MVs.
Our results are promising and suggest many avenues for future work.
In particular, we are interested in deeper exploration of the multiple MV setting.
There are many interesting design problems such as given storage constraints and throughput demands, optimize sampling ratios over all views.
Furthermore, there is an interesting challenge about queries that join mutliple sample MVs managed by \svc.
We are also interested in the possibility of sharing computation between MVs and maintenance on views derived from other views.
%Finally, our results suggest relatively a straight forward implementation of adaptive selection of the parameters in \svc such as the view sampling ratio and the outlier index threshold.

However, as with all sampling and approximation techniques, there are a few limitations which we summarize in this section.
\svc does not support views with ordering or ``top-k'' clauses, as our sampling assumes no ordering on the rows of the MV.
Furthermore, while there are a few key special cases (equality joins) arbitrary joins do not commute with sampling, making \svc ineffecient.
\svc also requires the maintenance strategy to be parameterized in terms of relational algebra which may not always be possible if that is a black blox.
In terms of queries on the view, as with previous work in AQP, sampling is best suited for aggregate queries.
While we proposed a technique that can give answers for some select queries, in general, \svc will give poor results for very selective queries.
The limitations of sampling have been discussed extensively in \cite{DBLP:journals/ftdb/CormodeGHJ12}.

\vspace{-1em}
\section{Conclusion}\label{conclusion}
\vspace{-.3em}
%In this paper, we propose a new framework, \svc, to address the staleness problem in materialized views.
Materialized view maintenance is often expensive, and in practice, immediate view maintenance is avoided due to its costs.
However, this leads to stale materialized views which have incorrect, missing, and superfluous rows.
In this work, we formalize the problem of staleness and view maintenance as a data cleaning problem.

\svc uses a sample-based data cleaning approach to get accurate query results that reflect the most recent data for a greatly reduced computational cost.
To achieve this, we significantly extended our prior work in data cleaning, SampleClean \cite{wang1999sample}, for efficient cleaning of stale MVs. 
This included processing a wider set of aggregate queries, handling missing data errors, and proving for which queries optimality of the estimates hold.
Another sigificant contribution of \svc is our outlier indexing approach which reduces the sensitivity of sampling to skewed data distributions.
We presented both empirical and theoretical results showing that our sample data cleaning approach is significantly less expensive than full view maintenance for a large class of materialized views, while still providing accurate aggregate query answers that reflect the most recent data.
We evaluate \svc on a real dataset of server logs from Conviva and the TPCD benchmark dataset, and our experiments confirm our theoretical results: (1) cleaning an MV sample is more efficient than full view maintenance, (2) the corrected results are more accurate than using the stale MV, and (3) SVC can be efficiently integrated with deferred maintenance leading to increased query accuracy. 






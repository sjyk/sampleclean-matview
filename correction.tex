\section{Correction Approximate Query Processing}

In this section, we will extend the NormalizedSC algorithm to estimate
corrections for queries on stale views. The key challenge is to estimate
$\epsilon$ such that:
\[
f(\textbf{V}_{T}^{'})=f(\textbf{V}_{T})+\epsilon
\]



\subsection{Select-Project and Foreign-Key Join Views}

We will first derive the exact value for $\epsilon$ without sampling.
Since we only consider a model were records are inserted into the
base tables, for these two categories of views $\textbf{V}_{T}\subseteq\textbf{V}_{T}^{'}$.
The row differences between $\textbf{V}_{T}$ and $\textbf{V}_{T}^{'}$
are completely represented by the delta table $\Delta\textbf{V}$;
that is rows will only be inserted into the views. The aggregation
functions SUM, COUNT, AVG, and VAR are special as they can be expressed
as summations. The summative forms lead to the following insight:
\[
f(\textbf{V}_{T}^{'})=f(\textbf{V}_{T})+\epsilon
\]
\[
f(\textbf{V}_{T}^{'})-f(\textbf{V}_{T})=\epsilon
\]


\[
c\cdot f(\Delta\textbf{V})=\epsilon
\]
Up-to a scaling constant c, $\epsilon$ is the aggregation function
applied to the delta table. 

\begin{center}
\begin{tabular}{|c|c|}
\hline 
Aggregation Query F & Scaling Constant c\tabularnewline
\hline 
\hline 
SUM & 1\tabularnewline
\hline 
COUNT & 1\tabularnewline
\hline 
AVG & $\frac{|\Delta V|}{|\Delta V|+|V|}$\tabularnewline
\hline 
\end{tabular}
\par\end{center}


\subsubsection{Sampling the Delta View}

We can extend this theory to a simple random sample $S_{\Delta V}$
of the delta view $S_{\Delta V}\subseteq\Delta\textbf{V}$. Recall
that a simple random sample is uniform sample where every row $r\in\Delta V$
is in $S_{\Delta V}$ with equal probability $p$. For Select-Project
and Foreign-Key Join Views, this means we have to take a sample of
the updates and then apply the view definition to the sample of the
updates. Formally, for every record u inserted into the table, with
probability $p$, we include it in the sample $S$. Then, we take
the sample updates $S$ and apply the view definition forming $S_{\Delta V}$.
Therefore,
\[
c\cdot f(S_{\Delta V})\approx\epsilon
\]
Due to the summative forms and uniform sampling, we can apply the
Central Limit Theorem to bound the approximation error. Sums of independent
random variables converge to a normal distribution, and further more
the expected value of $c\cdot f(S_{\Delta V})$ is $\epsilon$. 
\[
c\cdot f(S_{\Delta V})\sim N(\epsilon,\frac{\sigma_{diff}^{2}}{k})
\]
\[
f(\textbf{V}_{T})+c\cdot f(S_{\Delta V})\sim N(f(\textbf{V}_{T}^{'}),\frac{\sigma_{diff}^{2}}{k})
\]
$\sigma_{diff}^{2}$ is an interesting parameter as it quantifies
the variance of the delta view. In Section ?, we will analyze this
parameter and describe how the statistics of the updates affect the
estimate accuracy.


\subsection{Aggregation Views}

Applying the view definition to the updates is not enough information
to calculate $\epsilon$ in aggregation views. Consider the following
example view and query pair:

\begin{lstlisting}
View1 := SELECT col2, MAX(col1) as col1_max
FROM table 
GROUP BY col2
\end{lstlisting}


\begin{lstlisting}
Query1 := SELECT AVG(col1_max) 
FROM View1 
\end{lstlisting}


\begin{center}
\begin{tabular}{|c|c|}
\hline 
col1 & col2\tabularnewline
\hline 
\hline 
3 & 1\tabularnewline
\hline 
6 & 1\tabularnewline
\hline 
2 & 2\tabularnewline
\hline 
\end{tabular} %
\begin{tabular}{|c|c|}
\hline 
col2 & col1\_max\tabularnewline
\hline 
\hline 
1 & 6\tabularnewline
\hline 
2 & 2\tabularnewline
\hline 
\end{tabular}
\par\end{center}

Suppose records are inserted into \textbf{table} and we can apply
the the definition View1 to set of inserted records:

\begin{center}
\begin{tabular}{|c|c|}
\hline 
col1 & col2\tabularnewline
\hline 
\hline 
3 & 1\tabularnewline
\hline 
6 & 1\tabularnewline
\hline 
2 & 2\tabularnewline
\hline 
\textbf{2} & \textbf{1}\tabularnewline
\hline 
\end{tabular} %
\begin{tabular}{|c|c|}
\hline 
col2 & col1\_max\tabularnewline
\hline 
\hline 
2 & 1\tabularnewline
\hline 
\end{tabular}
\par\end{center}

However, we see that when we perform the merge operation, the updated
View1 remains the same, thus the $\epsilon$ for Query1 is 0, even
though the delta table has non-zero rows:

\begin{center}
\begin{tabular}{|c|c|}
\hline 
col2 & col1\_max\tabularnewline
\hline 
\hline 
1 & 6\tabularnewline
\hline 
2 & 2\tabularnewline
\hline 
\end{tabular}
\par\end{center}

The key point is that the merge operation depends on the aggregations
in view definition, and we need to know how these aggregates change
after the merge to estimate $\epsilon$. Let $\textbf{W}$ be the
join of up-to-date view $\textbf{V}_{T}^{'}$ and the old view $\textbf{V}_{T}$
on the group-by key. To make the example above more interesting we
can insert a few more records, $\textbf{W}$ would be:

\begin{center}
\begin{tabular}{|c|c|}
\hline 
col1 & col2\tabularnewline
\hline 
\hline 
3 & 1\tabularnewline
\hline 
6 & 1\tabularnewline
\hline 
2 & 2\tabularnewline
\hline 
\textbf{2} & \textbf{1}\tabularnewline
\hline 
\textbf{4} & \textbf{2}\tabularnewline
\hline 
\end{tabular} %
\begin{tabular}{|c|c|c|}
\hline 
col2 & col1\_max\_new & col1\_max\_old\tabularnewline
\hline 
\hline 
1 & 6 & 6\tabularnewline
\hline 
2 & 4 & 2\tabularnewline
\hline 
\end{tabular}
\par\end{center}

The value of $\epsilon$ for Query1 would be the avg query applied
to the difference:

\begin{lstlisting}
QueryEpsilon1 := SELECT AVG(col1_max_new-col1_max_old) 
FROM W
\end{lstlisting}



\subsubsection{Sampling the Merged View}

We realize that in the merged view each GROUP BY key is unique, and
thus, to sample the merged view we have to sample by GROUP BY keys
in the inserted records. For each inserted record we apply a hash
to the cols in the GROUP BY clause, and then we take the result of
the hash modulo a sampling ratio to sample the table. The result is
that we ensure that every record with the same group by key is either
fully in the sample or not, thus none of the rows in the delta view
are approximate. Then, we merge this sample delta view with the old
view to get the set of differences.
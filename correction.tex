\section{Correction Query Processing}
In this section, we will describe the how to calculate a 
correction for stale queries.
Suppose we have an aggregation query $f$ and let $\textbf{V}_{T}^{'}$ be the up-to-date view
and $\textbf{V}_{T}$ be the old view. 
The query $f$ is stale with error $\epsilon$ if:
\[
f(\textbf{V}_{T}^{'})=f(\textbf{V}_{T})+\epsilon
\]
Before we discuss how sampling can scale this process, we will discuss how to correct
queries on the full data.

\subsection{Model for Aggregate Queries on Views}
What is special about SUM, COUNT, AVG, and VAR aggregate functions
is that upto proportionality constant, they can be expressed as a mean-value calculation.
For example, a SUM is just the mean value times the dataset size, a COUNT 
is the mean occurance rate times the dataset size, and VAR is mean squared 
deviation.
Let $N$ be the number of tuples in the view $V$. 
These queries can also have predicates so we have to incorporate that
as an indicator function (true = 1, false = 0) that skips a tuple in the aggregation if the predicate is false. 
When we couple these queries with predicates, we can express them in the 
following way:
\[
\forall v_i \in V \text{ : } f(V)= \frac{1}{N} \sum_i^N \phi(v_i) \cdot predicate(i)
\]

We define $\phi$ in the following way:
\begin{center}
\begin{tabular}{|c|c|}
\hline 
Aggregation Query & $\phi(v_i)$\tabularnewline
\hline 
\hline 
SUM & $N \cdot v_i$\tabularnewline
\hline 
COUNT & $N$\tabularnewline
\hline 
AVG & $\frac{N}{\sum_i^N predicate(i)} \cdot v_i$\tabularnewline
\hline 
\end{tabular}
\par\end{center}

\subsection{Corrections For Select-Project and Foreign-Key Join Views}
With the model for aggregate queries described above, we can
first derive the exact value for $\epsilon$ without sampling.
Since we only consider a model were records are inserted into the
base tables, for these two categories of views $\textbf{V}_{T}\subseteq\textbf{V}_{T}^{'}$.
The row differences between $\textbf{V}_{T}$ and $\textbf{V}_{T}^{'}$
are completely represented by the delta table $\Delta\textbf{V}$;
that is rows will only be inserted into the views. 
Since the aggregate queries are in the form of means, we notice that we 
can exploit the associativity of summations:
\[
f(\textbf{V}_{T}^{'})=f(\textbf{V}_{T})+\epsilon
\]
\[
f(\textbf{V}_{T}^{'})-f(\textbf{V}_{T})=\epsilon
\]
Upto a scaling constant c, $\epsilon$ is the aggregation function
applied to the delta table. 
\[
c\cdot f(\Delta\textbf{V})=\epsilon
\]

\begin{center}
\begin{tabular}{|c|c|}
\hline 
Aggregation Query & Scaling Constant c\tabularnewline
\hline 
\hline 
SUM & 1\tabularnewline
\hline 
COUNT & 1\tabularnewline
\hline 
AVG & $\frac{|\Delta V|}{|\Delta V|+|V|}$\tabularnewline
\hline 
\end{tabular}
\par\end{center}

\subsubsection{Example Query Processing}
Recall, our example dataset of video streaming logs and the example
selection view:
\begin{lstlisting}
View1 := SELECT * FROM Log 
WHERE userAgent 
LIKE '%Mozilla%'
\end{lstlisting}
This query filters out the log records that came from browsers with the Mozilla tag.
Let us assume that our old materialized view has 1M rows, and 
we receive an update of 500,000 new rows with the mozilla tag.
Now we want to answer the following query:
\begin{lstlisting}
SELECT avg(responseTime) 
FROM View1
\end{lstlisting}
On the inserted 500,000 rows, we can run the query and call the result $r_{delta}$.
Then, we apply the scaling constant c to the result to convert this into an $\epsilon$ and get $r_{delta}\frac{500000}{500000+1000000}$, which 
equals $\frac{r_{delta}}{3}$.
Therefore, the up-to-date result is:
\[r_{old} + \epsilon = r_{old} +\frac{r_{delta}}{3}\].

\subsection{Aggregation Views}
The delta view is not enough information to calculate $\epsilon$ in aggregation views. 
Consider the following example view which we described in the last section:
\begin{lstlisting}
View2 := SELECT videoID, 
max(responseTime) AS maxResponseTime 
FROM Log
GROUP BY videoID;
\end{lstlisting}
Now we want to correct the following stale query.
\begin{lstlisting}
SELECT COUNT(1) 
FROM View2 
WHERE maxResponseTime > 100ms;
\end{lstlisting}
Now suppose, View2 looks like this:

\begin{tabular}{|c|c|}
\hline 
videoId & maxResponseTime\tabularnewline
\hline 
\hline 
125 & 99\tabularnewline
\hline 
6212 & 160\tabularnewline
\hline 
222 & 145\tabularnewline
\hline 
\end{tabular}

We may get a delta table for this view of the form:

\begin{tabular}{|c|c|}
\hline 
videoId & maxResponseTime\_max\tabularnewline
\hline 
\hline 
125 & 96\tabularnewline
\hline 
\end{tabular}

However, we see that when we perform the refresh operation, the updated
View2 remains the same since 96 < 99. 
Thus the $\epsilon$ for query is 0, even though the delta table has non-zero rows. 
The key point is that the refresh operation depends on the values in the view, 
and we need to know how these aggregates change
after the refresh to estimate $\epsilon$. 
Let $\textbf{W}$ be the join of up-to-date view $\textbf{V}_{T}^{'}$ and the old view $\textbf{V}_{T}$
on the group-by key:

\begin{tabular}{|c|c|c|}
\hline 
videoId & maxResponseTime\_new & maxResponseTime\_old \tabularnewline
\hline 
\hline 
125 & 99 & 99\tabularnewline
\hline 
6212 & 160 & 160\tabularnewline
\hline 
222 & 145 & 145\tabularnewline
\hline 
\end{tabular}

However, an interesting about aggregation views is that they do not 
require scaling constant $c$ as in the the other two categories of views.
This is because we refresh the delta view; inferring a correction from the
entire view rather than just the updates.

\subsubsection{Example Query Processing}
We can make the example in the previous section more interesting to illustrate the query processing steps.
Suppose our delta table was the following:

\begin{tabular}{|c|c|}
\hline 
videoId & maxResponseTime\_max\tabularnewline
\hline 
\hline 
125 & 96\tabularnewline
\hline 
1336 & 214\tabularnewline
\hline 
\end{tabular}

Then the joined result would be:

\begin{tabular}{|c|c|c|}
\hline 
videoId & maxResponseTime\_new & maxResponseTime\_old \tabularnewline
\hline 
\hline 
125 & 99 & 99\tabularnewline
\hline 
6212 & 160 & 160\tabularnewline
\hline 
222 & 145 & 145\tabularnewline
\hline 
1336 & 214 & NULL\tabularnewline
\hline 
\end{tabular}

We can transform the example:

\begin{lstlisting}
SELECT COUNT(1) 
FROM View2 
WHERE maxResponseTime > 100ms;
\end{lstlisting}

in terms of SQL case statements that evaluate a boolean to 1 or 0 if true or false/NULL.
\begin{lstlisting}
SELECT (maxResponseTime_new > 100ms) 
- (maxResponseTime_old > 100ms)

FROM Joined_View2 
\end{lstlisting}
The result of this query on the example is 1 which is a correction to the stale count 
of videos with a max response time of greater than 100ms.
